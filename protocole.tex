\documentclass{report}
\usepackage[T1]{fontenc}
\usepackage[utf8]{inputenc}
\usepackage[francais]{babel}
\usepackage{amsmath}
\usepackage{graphicx}
%\usepackage[backend=biber,style=authoryear,bibencoding=utf8]{biblatex}
%\usepackage[colorlinks,linkcolor=blue]{hyperref}
%
%\addbibresource{biblio.bib}
\begin{document}
\section{Protocole d'étirement des cellules}

\begin{itemize}
\item Bien nettoyer toutes les pièces à l'alcool 70\%
\item Monter une lamelle de verre 30mm de diamètre enroulée dans du téflon entre les deux parties de la cuve
\item Remplir la cuve avec le milieu qui était dans le puit où était la lamelle qu'on veut observer
\item Ajouter 1,5pour cent d'HEPES 1mM (donc 75micro l pour 5 ml)
\item Monter la lamelle cellules vers le bas sur le support entre les deux anneaux de téflon et mettre rapidement dans la cuve
\item S'assurer qu'il n'y a pas de bulles sous la lamelle
\item prélever 1ml dans la cuve et le mettre au-dessus de la lamelle (pour réduire le frottement avec le plot
\item Recouvrir avec le plot (tourner le pas de vis à l'envers jusqu'au clac d'enclenchement)
\item NE PAS ETIRER IMMEDIATEMENT
\item Prendre des images des cellules non étirées dans un dossier spécifique
\item étirer (3 tours complets après l'alignement de la pièce du dessus et de la cuve)NOTER L'HEURE
\item Chercher des cellules MRTF-A GFP et enregistrer leur position dans les différents protocoles 'zones'. Toutes les 5 minutes, reprendre des images des zones déjà repérées. 
\item Ordre des couleurs : Rouge profond, GFP, DAPI, BF
\item Après 30 minutes d'étirement, arrêter de chercher de nouvelles zones et tourner entre les zones déjà repérées. 
\end{itemize}



\end{document}
