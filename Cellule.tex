\documentclass{report}
\usepackage[T1]{fontenc}
\usepackage[utf8]{inputenc}
\usepackage[francais]{babel}
\usepackage{amsmath}
\usepackage{graphicx}
\usepackage[backend=biber,style=authoryear,bibencoding=utf8]{biblatex}
\usepackage[colorlinks,linkcolor=blue]{hyperref}

\addbibresource{biblio.bib}

\begin{document}
\chapter{La cellule et son environnement mécanique}


La cellule est l'unité de base des êtres vivants. Elle peut extraire de l'énergie du milieu extérieur afin de se maintenir dans un état organisé et complexe et de se reproduire pour donner naissance à d'autres cellules par la division cellulaire. 

Une cellule est séparée du milieu extérieur par une membrane. Elle contient son code génétique sous la forme d'ADN, elle le duplique et le transmet lors de ses divisions. 

Les cellules sont séparées en deux grands groupes en fonction de l'état de leur ADN : les procaryotes et les eucaryotes. 
L'ADN des procaryotes est libre dans la cellule, il est souvent constitué d'un seul chromosome circulaire. 
Au contraire, l'ADN des eucaryotes est confiné dans un compartiment spécial, le noyau. 

Chez les eucaryotes comme chez les procaryotes, il existe les organismes vivants peuvent être composés d'une ou de plusieurs cellules. 
Les plus grands organismes, comme les plantes ou les animaux, peuvent être composées d'un très grand nombre de cellules (de l'ordre de $10^{14}$) qui possèdent le même génome mais ont des phénotypes très divers. 

\section{Organisation de la cellule eucaryote}

\subsection{Les compartiments}

Les cellules eucaryotes sont composées de plusieurs compartiments aux fonctions spécifiques à l'intérieur d'une membrane plasmique. Le noyau est le plus gros de ces compartiments, il renferme l'ADN organisé sous la forme de chromosomes et est le lieu de la transcription de l'ADN en ARN. 
Le réticulum endoplasmique rugueux est le siège de la traduction de l'ARN en protéines. 
L'appareil de Golgi est le lieu de transformation finale des protéines. 
Les mitochondries sont les unités de production d'énergie de la cellule : elles produisent l'Adénosine Triphosphate (ATP), qui sera transformée en Adénosine Diphosphate (ADP) en libérant de l'énergie. Les mitochondries sont d'anciennes bactéries devenues symbiotiques des cellules eucaryotes, elles possèdent leur propre ADN. 


\subsubsection{Le noyau}

Le noyau sépare le matériel génétique du reste du milieu cellulaire. Cependant, bien d'autres molécules que l'ADN sont présentes dans le noyau, et il s'y passe d'autres choses que la transcription ou la duplication de l'ADN. 

L'ADN est une molécule constituée d'un enchaînement de nucléotides composés d'une base azotée, d'un sucre et d'un groupement phosphates. Il existe quatre bases azotées possibles : adénine, guanine, thymine et cytosine. Leur enchaînement va être à la base du code génétique.
L'enchaînement des nucléotides forme une double hélice dans laquelle les nucléotides se correspondent en deux paires : Adénosine et Thymine, Cytosine et Guanine. 

Le génome des être vivants peut comporter typiquement du million au milliard de paires de base d'ADN, la distance entre deux bases étant de 0,34nm. Chez l'homme, on compte environ 3,2 milliards de paires de bases, ce qui correspond à une longueur d'ADN de l'ordre du mètre, qui doit être stockée dans le noyau cellulaire d'un diamètre de 5 à 7 microns. 
On conçoit alors qu'une organisation spécifique de l'ADN dans le noyau soit nécessaire pour faire tenir une molécule aussi grande dans un compartiment aussi étroit. 

L'ADN est enroulé autour de protéines appelées histones comme du fil autour d'une bobine, formant un nucléosome. Ces nucléosomes empilés forment une structure bien plus compacte que l'ADN libre, appelée la chromatine. Certains acides aminés qui composent les histones peuvent subir des réactions chimiques comme l'acétylation ou la méthylation, qui ont un rôle dans la régulation de la transcription du génome. L'acétylation des histones détermine l'état de la chromatine : l'hétérochromatine est la forme compacte où l'ADN ne peut pas être transcrit, l'euchromatine est la forme plus étendue dans laquelle l'ADN est accessible pour la transcription. 

Le noyau est séparé du reste du cytoplasme par une membrane formée de deux bicouches lipidiques. Elle est percées de trous appelés pores nucléaires, par lesquels les molécules peuvent entrer et sortir du noyau. Les protéines de taille inférieure à 40 kDa peuvent passer par diffusion passive par les pores nucléaires. 

Les protéines de plus grande taille doivent faire appel à des transporteurs spécifiques pour aller d'un côté de la membrane à l'autre. 
Les importines vont localiser sur la protéine à importer un signal de localisation nucléaire (NLS) et se lier à elle par ce biais. Le couple importine-cargo va diffuser à travers le pore nucléaire. Une fois dans le noyau, l'importine va se lier à la RanGTP et se dissocier du cargo, qui est libéré dans le nucléoplasme. L'importine est alors à nouveau exportée du noyau et la GTP hydrolysée en GDP. 
De même, une protéine possédant une sequence d'export nucléaire (NES) sera liée à une exportine-GTP, et le couple diffusera vers le cytoplasme. Une fois dans le cytoplasme, la GTP est hydrolysée en GDP et le cargo relargué. L'exportine revient dans le noyau par diffusion. 

Par exemple, l'actine, bien que de taille 42kDa, à la limite de la diffusion passive, est importée de manière active par l'importine 9 et exportée par l'exportine 6. 



\end{document}