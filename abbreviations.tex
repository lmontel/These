\chapter*{Liste des abbréviations}

ABP : Actin Binding Protein, protéine se liant à l'actine\\
ADN : Acide Désoxyribonucléique, support de l'information génétique\\
ADP : Adénosine Di-Phosphate\\
AFM : Atomic Force Microscopy, Microscope à force atomique\\
AOTF : Acousto-optical tunable filter, Modulateur acousto-optique\\
Arp2/3 : Complexe formé des Actin Related Proteins 2 et 3 entre autres\\
ATCC : American Type Culture Collection\\
ATP : Adénosine Tri-Phosphate\\
BSA : Bovine Serum Albumine\\
BSAC : Basic, Sap and Coiled-Coil, autre nom pour la MRTF-A murine\\
C : localisation majoritairement Cytoplasmique\\
CarG : séquence CC(A/T)$_6$GG sur l'ADN où peut se fixer SRF\\
ChIP seg : séquençage par immunoprécipitation de chromatine\\
Cox2 : Cyclo-oxygénase 2\\
Cy3 : Cyanine fluorescente dans le rouge\\
C2C12 : lignée de myoblastes murins immortalisés\\
DAPI :  4',6'-diamidino-2-phénylindole, marqueur de l'ADN fluorescent dans le bleu\\
DMEM : Dulbecco's Modified Eagle Medium, milieu de culture cellulaire\\
DN : Dominant Négatif, version déficiente de la protéine qui interfère avec la version saine\\
DNase : Désoxyribonucléase I, utilisée comme marqueurs de l'actine monomérique\\
Ddx19 : protéine liée à l'import de MRTF-A\\
Ecad : cadhérine épithéliale\\
EDTA : Éthylène Diamine Tétra-Acétique, utilisé comme chélateur des ions Calcium\\
EMT : Epithélial to Mesenchymal Transition, Transition d'un type épithélial à un type mésenchymateux\\
ERK 1/2 : Extracellular signal-regulated kinases, kinases phosphorylant MRTF-A\\
FRET : Fluorescence Resonance Energy Transfer, technique de microscopie combinant deux fluorophores très proches\\
GFP : Green Fluorescent Protein, protéine fluorescent dans le vert et qui peut être ajoutée à n'importe quelle protéine à visualiser\\
GTP : Guanosine Tri-Phosphate\\
H : localisation majoritairement Homogène\\
HS : Horse Serum\\
HEPES :  acide 4-(2-hydroxyéthyl)-1-pipérazine éthane sulfonique, solution tampon\\
IL4/6 : Interleukines 4 et 6\\
ILK : Integrin-linked kinase\\
KO : Knock Out, désactivation d'un gène\\
LIMK : LIM domain kinase, kinase phosphorylant la cofiline\\
MAL : Magekaryocytic Acute Leukemia, autre nom pour MRTF-A\\
MAPK : Mitogen-activated protein kinases, qui forment une voie de signalisation activant SRF et ERK1/2\\
mDia : mammalian Diaphanous, une famille de formines\\
MEC : Matrice Extra-Cellulaire\\
MICAL : protéines capables d'oxyder l'actine\\
MKL : MegaKacyocytic Leukemia, autre nom pour MRTF-A\\
MRTF : Myocardin-Related Transcription Factor, famille de protéines objet du chapitre 3\\
MyoD : Facteur de différenciation musculaire
N : localisation majoritairement Nucléaire\\
NES : Nuclear Export Signal, séquence d'acides aminés permettant l'export du noyau vers le cytoplasme\\
NLS : Nuclear Localisation Signal, séquence d'acides mainés permettent l'import du noyau vers le cytoplasme\\
NM1 : Nuclear myosin 1\\
OTT-MAL : version fusionnée de MRTF-A à l'origine de leucémies\\
PBS :  Phosphate buffered saline, solution tampon\\
PDMS : Polydimethylsiloxane, polymère visqueux pouvant être réticulé en gel\\
PFA : Paraformaldéhyde, pour la fixation\\
Pol I/II/III : ARN polymérases\\
RhoA : Ras homolog gene family, member A, petite GTPase centrale dans l'activation de MRTF-A\\
ROCK : rho-associated, coiled-coil-containing protein kinase, phosphorylant la LIMK\\
RPEL : séquence d'acide aminés permettent de se lier à un ou des monomères d'actine (3 RPEL pour 5 actines)\\
SiRNA : Small interfering RNA, petits ARN qui permettent d'empêcher la synthèse d'une protéine donnée\\
SRE : Serum Response Element, promoteur des cibles de SRF\\
SRF : Serum Response Factor, facteur de transcription \\
SVF : Serum de Veau F\oe tal, facteur de croissance\\
TAD : Transactivation Domain, séquence d'acide amminés d'un facteur de transcription qui peut activer la trancription\\
T$\beta$4 : Thymosine $\beta$ 4, protéine séquestrant les monomères d'actine\\
TCF : Ternary Complex Factors, famille de protéines activant SRF\\
TGF $\beta$ : Transforming growth factor $\beta$\\
TNF $\alpha$ : Tumor necrosis factor $\alpha$\\
UV : Ultra-Violet\\
WT : Wild-Type, protéine sauvage\\