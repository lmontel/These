\documentclass{report}
\usepackage[T1]{fontenc}
\usepackage[utf8]{inputenc}
\usepackage[francais]{babel}
\usepackage{amsmath}
\usepackage{graphicx}
\usepackage[backend=biber,style=authoryear,bibencoding=utf8]{biblatex}
\usepackage[colorlinks,linkcolor=blue]{hyperref}

<<<<<<< HEAD
\addbibresource{biblio.bib}

\begin{document}
\chapter{La cellule et son environnement mécanique}


La cellule est l'unité de base des êtres vivants. Elle peut extraire de l'énergie du milieu extérieur afin de se maintenir dans un état organisé et complexe et de se reproduire pour donner naissance à d'autres cellules par la division cellulaire. 

Une cellule est séparée du milieu extérieur par une membrane. Elle contient son code génétique sous la forme d'ADN, elle le duplique et le transmet lors de ses divisions. 

Les cellules sont séparées en deux grands groupes en fonction de l'état de leur ADN : les procaryotes et les eucaryotes. 
L'ADN des procaryotes est libre dans la cellule, il est souvent constitué d'un seul chromosome circulaire. 
Au contraire, l'ADN des eucaryotes est confiné dans un compartiment spécial, le noyau. 

Chez les eucaryotes comme chez les procaryotes, il existe les organismes vivants peuvent être composés d'une ou de plusieurs cellules. 
Les plus grands organismes, comme les plantes ou les animaux, peuvent être composées d'un très grand nombre de cellules (de l'ordre de $10^{14}$) qui possèdent le même génome mais ont des phénotypes très divers. 

\section{Organisation de la cellule eucaryote}

\subsection{Les compartiments}

Les cellules eucaryotes sont composées de plusieurs compartiments aux fonctions spécifiques à l'intérieur d'une membrane plasmique. Le noyau est le plus gros de ces compartiments, il renferme l'ADN organisé sous la forme de chromosomes et est le lieu de la transcription de l'ADN en ARN. 
Le réticulum endoplasmique rugueux est le siège de la traduction de l'ARN en protéines. 
L'appareil de Golgi est le lieu de transformation finale des protéines. 
Les mitochondries sont les unités de production d'énergie de la cellule : elles produisent l'Adénosine Triphosphate (ATP), qui sera transformée en Adénosine Diphosphate (ADP) en libérant de l'énergie. Les mitochondries sont d'anciennes bactéries devenues symbiotiques des cellules eucaryotes, elles possèdent leur propre ADN. 


\subsubsection{Le noyau}

Le noyau sépare le matériel génétique du reste du milieu cellulaire. Cependant, bien d'autres molécules que l'ADN sont présentes dans le noyau, et il s'y passe d'autres choses que la transcription ou la duplication de l'ADN. 

L'ADN est une molécule constituée d'un enchaînement de nucléotides composés d'une base azotée, d'un sucre et d'un groupement phosphates. Il existe quatre bases azotées possibles : adénine, guanine, thymine et cytosine. Leur enchaînement va être à la base du code génétique.
L'enchaînement des nucléotides forme une double hélice dans laquelle les nucléotides se correspondent en deux paires : Adénosine et Thymine, Cytosine et Guanine. 

Le génome des être vivants peut comporter typiquement du million au milliard de paires de base d'ADN, la distance entre deux bases étant de 0,34nm. Chez l'homme, on compte environ 3,2 milliards de paires de bases, ce qui correspond à une longueur d'ADN de l'ordre du mètre, qui doit être stockée dans le noyau cellulaire d'un diamètre de 5 à 7 microns. 
On conçoit alors qu'une organisation spécifique de l'ADN dans le noyau soit nécessaire pour faire tenir une molécule aussi grande dans un compartiment aussi étroit. 

L'ADN est enroulé autour de protéines appelées histones comme du fil autour d'une bobine, formant un nucléosome. Ces nucléosomes empilés forment une structure bien plus compacte que l'ADN libre, appelée la chromatine. Certains acides aminés qui composent les histones peuvent subir des réactions chimiques comme l'acétylation ou la méthylation, qui ont un rôle dans la régulation de la transcription du génome. L'acétylation des histones détermine l'état de la chromatine : l'hétérochromatine est la forme compacte où l'ADN ne peut pas être transcrit, l'euchromatine est la forme plus étendue dans laquelle l'ADN est accessible pour la transcription. 

Le noyau est séparé du reste du cytoplasme par une membrane formée de deux bicouches lipidiques. Elle est percées de trous appelés pores nucléaires, par lesquels les molécules peuvent entrer et sortir du noyau. Les protéines de taille inférieure à 40 kDa peuvent passer par diffusion passive par les pores nucléaires. 

Les protéines de plus grande taille doivent faire appel à des transporteurs spécifiques pour aller d'un côté de la membrane à l'autre. 
Les importines vont localiser sur la protéine à importer un signal de localisation nucléaire (NLS) et se lier à elle par ce biais. Le couple importine-cargo va diffuser à travers le pore nucléaire. Une fois dans le noyau, l'importine va se lier à la RanGTP et se dissocier du cargo, qui est libéré dans le nucléoplasme. L'importine est alors à nouveau exportée du noyau et la GTP hydrolysée en GDP. 
De même, une protéine possédant une sequence d'export nucléaire (NES) sera liée à une exportine-GTP, et le couple diffusera vers le cytoplasme. Une fois dans le cytoplasme, la GTP est hydrolysée en GDP et le cargo relargué. L'exportine revient dans le noyau par diffusion. 

Par exemple, l'actine, bien que de taille 42kDa, à la limite de la diffusion passive, est importée de manière active par l'importine 9 et exportée par l'exportine 6. 

=======
\begin{document}

\section{•}

Le génome d'un organisme contient toute l'information nécessaire à la reconstitution de l'organisme entier. Cependant, il est évident dans un organisme pluricellulaire que si toutes les cellules contiennent le même génome, elles ne l'expriment pas de la même manière. C'est également le cas chez des être unicellulaires : des bactéries ou des levures ayant le même génome ne vont pas l'exprimer de la même manière selon les conditions extérieures. 

Les parties codant directement pour des protéines ne représentent qu'une toute petite partie de l'ADN. Autour des séquences codantes, le reste du génome permet de déterminer quelles protéines doivent être synthétisées, et dans quelles quantités. 

\subsection{Des gènes aux protéines}

Le chemin d'un gène à une protéine fonctionnelle passe par trois étapes principale : la transcription de l'ADN en ARN, la maturation de l'ARN et la traduction de l'ARN en protéine 

\subsubsection{La transcription}

L'information génétique est stockée dans l'ADN, pour y être protégée et transmise d'une génération à l'autre. Les quatre bases de l'ADN fonctionnent par paires, et grâce à ce mécanisme un brin d'ADN peut être complété par sa séquence miroir. 

Lorsque toutes les conditions d'expression sont réunies, l'ARN polymérase peut se fixer sur l'ADN au site de début de transcription. Le complexe ouvre l'ADN, et permet aux bases de l'ARN, Adénine, Uracile, Guanine et Cytosine, de compléter les bases de l'ADN, respectivement Thymosine, Adénine, Cytosine et Guanine. L'ARN polymérase progresse ainsi jusqu'à rencontrer une séquence terminateur sur l'ADN. 

À l'issu de cette étape, un ARN a été transcrit directement à partir de la séquence codante. 

\subsubsection{La maturation de l'ARN}

A l'issu de sa transcription, l'ARN subit plusieurs transformations. Des éléments sont ajoutés à ses deux extrémités, pour sa stabilité et sa reconnaissance par les ribosomes. 
L'ARN transcrit comprend deux types de séquences, les introns et les exons. Seuls les exons contiennent l'information des acides aminés pour coder la protéine. 
L'ARN va subir une étape d'épissage : les introns sont retirés de l'ARN jusqu'à ce qu'il ne reste que l'enchaînement des exons. Lors de cette étape, tous les exons peuvent être réassemblés, ou seulement une partie d'entre eux. Les protéines synthétisées par ces reconstitutions différentes seront différentes : on parle d'épissage alternatif. 
Un même gène peut donc coder pour des protéines différentes grâce au jeu de l'épissage. Chez la drosophile par exemple, un unique gène code pour autant de protéines différentes que tout le reste du génome. 

Après sa maturation, l'ARN messager est exporté du noyau pour se diriger vers le réticulum endoplasmique.

Tous les ARN n'ont pas vocation à être traduits en protéines : certains vont accomplir leur fonction sous cette forme. C'est le cas des éléments de la machine de traduction (ARN de transfert et ARN du ribosome), ou des micro-ARN, qui vont participer à la régulation de l'expression des gènes en aval de la transcription.

\subsubsection{La traduction}

Le code génétique est la correspondance entre des triplets de nucléotides et les acides aminés qui composent les protéines. 
Dans le cytoplasme, les ribosomes et les ARN de transfert vont faire la correspondance entre les nucléotides de l'ARN messager et les acides aminés, et assembler la suite des acides aminés jusqu'à atteindre un codon stop. 


\subsection{Régulation de l'expression du génome}

\subsubsection{Réorganisation de la chromatine}

L'ADN est stocké dans la cellule sous une forme condensée, la chromatine. Il est enroulé autour de complexes de protéines appelées les histones. Selon leur état biochimique (acétylation, méthylation) les histones forment un enroulement plus ou moins compact de l'ADN. 
Sous sa forme la plus condensée, l'hétérochromatine, l'ADN n'est pas accessible pour être lu par les ARN polymérase, il ne peut pas être transcrit. Par exemple, chez les femelles mammifères, l'un des deux chromosomes X est désactivé afin de ne pas avoir deux fois plus de transcription des gènes portés par ce chromosome qu'un organisme mâle. Il est condensé définitivement sous forme d'hétérochromatine. 

La compaction de l'ADN dans l'hétérochromatine ou dans l'euchromatine détermine donc quels gènes sont accessibles pour être trancrits et quels gènes sont désactivés dans l'hétérochromatine. 
La régulation de cette organisation par la modification chimique des histones est donc la première étape de la régulation trancriptionnelle du génome. 

\subsubsection{Les facteurs de transcription}

L'ARN polymérase ne peut pas se lier seule de manière stable sur l'ADN pour initier la transcription. 
Les facteurs de transcription sont une famille de protéines qui ont pour rôle de se fixer sur l'ADN en amont de la séquence à transcrire pour contribuer à l'expression du gène ou au contraire pour la réprimer. 

Un facteur de transcription reconnaît une séquence spécifique sur l'ADN en amont du gène régulé, le promoteur. Les promoteurs en amont d'un gène vont déterminer quels facteurs de transcription vont être capables d'activer la transcription. Des gènes partageant le même promoteur vont être activés par les mêmes facteurs de transcription et répondre aux mêmes stimuli.  

Dans le développement des êtres pluricellulaires, la différenciation des cellules souches totipotentes de l'embryon en différents types de tissus va être orchestrée par l'activation de nombreux facteurs de transcription. 

Les facteurs de transcription vont également être impliqués dans les réponses d'une cellule ou d'un organisme à des signaux biologique provenant de cellules voisines (par l'intermédiaire des liaisons transmembranaires entre deux cellules, par une structure spécialisée comme un synapse ou par un signal paracrine) ou des tissus éloignés (hormones circulant dans le corps), mais aussi à des signaux environnementaux comme la température, le choc osmotique, les contraintes mécaniques, l'exposition à la lumière du soleil  \dots


Un facteur de transcription peut recruter d'autres protéines, comme des coactivateurs (ou des corépresseurs), ou des protéines qui vont changer localement l'état de compacité de la chromatine, afin de rendre le gène plus facilement ou plus difficilement accessible. 

Un facteur de transcription ou ses coactivateurs doivent être présents dans le noyau pour accomplir leur fonction. Le contrôle de leur localisation dans la cellule permet d'activer ou de désactiver un facteur de transcription. Par exemple, le récepteur des \oe strogènes est principalement présent dans le cytoplasme en l'absence d'hormone. En présence d'hormones, il est transporté dans le noyau où il peut se lier à l'ADN et activer la trancription de ses gènes cibles. 


Un facteur de transcription peut également être régulé par la phosphrylation ou celle de ses cofacteurs, ou par la présence d'un ligand. 
>>>>>>> origin/master


\end{document}