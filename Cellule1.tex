\documentclass{report}
\usepackage[T1]{fontenc}
\usepackage[utf8]{inputenc}
\usepackage[francais]{babel}
\usepackage{amsmath}
\usepackage{graphicx}
\usepackage[backend=biber,style=authoryear,bibencoding=utf8]{biblatex}
\usepackage[colorlinks,linkcolor=blue]{hyperref}


\addbibresource{biblio.bib}

\begin{document}
\chapter{La cellule et son environnement mécanique}


La cellule est l'unité de base des êtres vivants. Elle peut extraire de l'énergie du milieu extérieur afin de se maintenir dans un état organisé et de se reproduire pour donner naissance à d'autres cellules par la division cellulaire. 

Une cellule est séparée du milieu extérieur par une membrane. Elle contient son code génétique sous la forme d'ADN, elle le duplique et le transmet lors de ses divisions. 

Les cellules sont séparées en deux grands groupes en fonction de l'état de leur ADN : les procaryotes et les eucaryotes. 
L'ADN des procaryotes est libre dans la cellule, il est souvent constitué d'un seul chromosome circulaire. 
Au contraire, l'ADN des eucaryotes est confiné dans un compartiment spécial, le noyau. 

Chez les eucaryotes comme chez les procaryotes, il existe des organismes vivants pouvant être composés d'une ou de plusieurs cellules. 
Les plus grands organismes, comme les plantes ou les animaux, peuvent être composées d'un très grand nombre de cellules (de l'ordre de $10^{14}$) qui possèdent le même génome mais ont des phénotypes très divers. 

\section{Organisation de la cellule eucaryote}


Les cellules eucaryotes sont composées de plusieurs compartiments aux fonctions spécifiques à l'intérieur d'une membrane plasmique. Le noyau est le plus gros de ces compartiments, il renferme l'ADN organisé sous la forme de chromosomes et est le lieu de la transcription de l'ADN en ARN. 
Le réticulum endoplasmique rugueux est le siège de la traduction de l'ARN en protéines. 
L'appareil de Golgi est le lieu de transformation finale des protéines. 
Les mitochondries sont les unités de production d'énergie de la cellule : elles produisent l'Adénosine Triphosphate (ATP), qui sera transformée en Adénosine Diphosphate (ADP) en libérant de l'énergie. Les mitochondries sont d'anciennes bactéries devenues symbiotiques des cellules eucaryotes, elles possèdent leur propre ADN. 

\subsection{L'énergie dans la cellule}

\subsection{Le noyau}

Le noyau sépare le matériel génétique du reste du milieu cellulaire. Cependant, bien d'autres molécules que l'ADN sont présentes dans le noyau, et il s'y passe d'autres choses que la transcription ou la duplication de l'ADN. 

\subsubsection{ADN}

L'ADN est une molécule constituée d'un enchaînement de nucléotides composés d'une base azotée, d'un sucre et d'un groupement phosphates. Il existe quatre bases azotées possibles : Adénine , Thymine, Cytosine, Guanine. Leur enchaînement va être à la base du code génétique.
L'enchaînement des nucléotides forme une double hélice dans laquelle les nucléotides se correspondent en deux paires : Adénine et Thymine, Cytosine et Guanine. 

Le génome des être vivants peut comporter typiquement du million au milliard de paires de base d'ADN, la distance entre deux bases étant de 0,34nm. Chez l'homme, on compte environ 3,2 milliards de paires de bases, ce qui correspond à une longueur d'ADN de l'ordre du mètre, qui doit être stockée dans le noyau cellulaire d'un diamètre de 5 à 7 microns. 
On conçoit alors qu'une organisation spécifique de l'ADN dans le noyau soit nécessaire pour faire tenir une molécule aussi grande dans un compartiment aussi étroit. 

L'ADN est enroulé autour de protéines appelées histones comme du fil autour d'une bobine, formant une structure appelée nucléosome. Ces nucléosomes empilés forment une structure bien plus compacte que l'ADN libre, appelée la chromatine. Certains acides aminés qui composent les histones peuvent subir des réactions chimiques comme l'acétylation ou la méthylation, qui ont un rôle dans la régulation de la transcription du génome. L'acétylation des histones détermine l'état de la chromatine : l'hétérochromatine est la forme compacte où l'ADN ne peut pas être transcrit, l'euchromatine est la forme plus étendue dans laquelle l'ADN est accessible pour la transcription. 

\subsubsection{Transport nucléo-cytoplasmique}
Le noyau est séparé du reste du cytoplasme par une membrane formée de deux bicouches lipidiques. Elle est percée de trous appelés pores nucléaires, par lesquels les molécules peuvent entrer et sortir du noyau. Les protéines de taille inférieure à 40 kDa peuvent passer par diffusion passive par les pores nucléaires. 

Les protéines de plus grande taille doivent faire appel à des transporteurs spécifiques pour aller d'un côté de la membrane à l'autre. 
Les importines vont localiser sur la protéine à importer un signal de localisation nucléaire (NLS) et se lier à elle par ce biais. Le couple importine-cargo va diffuser à travers le pore nucléaire. 
Une fois dans le noyau, l'importine va se lier à la RanGTP et se dissocier du cargo, qui est libéré dans le nucléoplasme. L'importine est alors à nouveau exportée du noyau et la GTP hydrolysée en GDP. 
De même, une protéine possédant une sequence d'export nucléaire (NES) sera liée à une exportine-GTP, et le couple diffusera vers le cytoplasme. Une fois dans le cytoplasme, la GTP est hydrolysée en GDP et le cargo relargué. L'exportine revient dans le noyau par diffusion. 

Par exemple, l'actine, bien que de taille 42kDa, à la limite de la diffusion passive, est importée de manière active par l'importine 9 et exportée par l'exportine 6. 

\subsection{La membrane plasmique}

La membrane plasmique sépare le milieu intérieur de la cellule de l'extérieur. Elle est composée d'une bicouche de lipides amphiphiles dans laquelle sont enchassées des molécules transmembranaires qui permettent de réguler les échanges avec le milieu. 

\subsubsection{Transport transmembranaire}
La membrane assure le passage de molécules d'un côté à l'autre de manière active ou passive. 

Les lipides peuvent passer par diffusion à travers la barrière, c'est le cas par exemple des hormones stéroïdiennes comme les androgènes ou les \oe strogènes, dont les récepteurs sont à l'intérieur de la cellule et non sur la membrane.

Les ions comme le calcium, le potassium ou le magnésium sont transportés par des canaux spécifiques. Le mouvement des ions permet de polariser la membrane et de faire passer un signal électrique et est essentiel dans le fonctionnement du système nerveux et dans la contraction musculaire. Des canaux spécialisés, les aquaporines, laissent passer l'eau pour moduler la pression osmotique. Il existe une très grande variété de transporteurs d'espèces chimiques à travers la membrane. 

Des molécules peuvent également être transportées de l'intérieur vers l'extérieur par l'invagination d'une partie de la membrane en une petite bulle, la vésicule. Ce phénomène s'appelle l'endocytose. Inversement, des vésicules dans le milieu intracellulaire peuvent fusionner avec la membrane pour émettre leur contenu vers l'extérieur, c'est l'exocytose. Les cellules fabriquant la matrice extra-cellulaire utilisent l'exocytose pour excréter les protéines synthétisées. 

\subsubsection{Signalisation}

La liaison d'un récepteur membranaire à un ligand peut déclencher une activité enzymatique, une ouverture de canaux ioniques ou l'activité des protéines G (dont les petites GTPases). 
La famille des protéines à 7 segments transmembranaires, par exemple, est responsable de la détection des signaux visuels, olfactifs, gustatifs, inflammatoires \dots

\subsubsection{Adhésion}

La plupart des cellules font partie d'un tissu et adhèrent à d'autres cellules voisines et aux protéines de la matrice extra-cellulaire par l'intermédiaire de protéines transmembranaires. 

\subsubsection{Interactions cellules-cellules}

Les cellules se lient les unes aux autres pour se reconnaître, communiquer et former une structure complète. 

La super famille des immunoglobulines comprend des molécules d'adhésion (IgCAM), souvent spécifiques à un type cellulaire. Elle comprend également les complexes majeurs d'histocompatibilité, qui permettent au système immunitaire de reconnaître les cellules de son propre organisme. 

Les cellules échangent entre elles des espèces chimiques et des signaux électriques. 
Les jonctions communicantes permettent une communication directe entre deux cellules en contact. Les espèces chimiques peuvent passer librement de l'une à l'autre. Cela peut amplifier un signal hormonal, perçu par une cellule et transmis à travers ces jonctions aux voisines qui n'ont pas capté le signal directement. Elles peuvent également servir à transmettre très rapidement un signal électrique. 
Les synapses sont un système très élaboré de communication entre deux cellules par échange d'espèces chimiques (les neurotransmetteurs) par exocytose dans un espace inter-cellulaire réduit. 

Les jonctions serrées permettent de créer une étanchéité de part et d'autre d'une monocouche cellulaire. Les cellules épithéliales forment un feuillet continu lié par des jonctions serrées qui maintient la séparation entre le milieu extérieur (la lumière) et le milieu intérieur, par exemple au niveau de la peau, des muqueuses, de l'intérieur du tube digestif\dots

Les cadhérines forment une famille de protéines exprimées partout dans l'organisme à tous les stades de son développement. 
Il existe une trentaine de gènes de cadhérines, et encore plus de protéines exprimées grâce à l'épissage. Ces différentes protéines sont spécifiques selon le type cellulaire.
Les cadhérines de deux cellules voisines peuvent interagir par leur domaine extra-cellulaire pour former une jonction (interaction cis). Les cadhérines peuvent également interagir par leur domaine transmembranaire entre cadhérine d'une même membrane et former des amas (interaction trans). 
Les cadhérines permettent de maintenir l'intégrité mécanique d'un tissu de cellules en connectant les cytosquelettes de cellules voisines entre eux. Les cadhérines desmosomales lient les réseaux de filaments intermédiaires, alors que les cadhérines classiques lient les réseaux d'actine. 


\subsubsection{Interactions cellules-matrice}

La famille des intégrines est le principal médiateur des interactions cellule-substrat. Elles forment des hétérodimères entre une forme alpha (parmi 18) et une forme beta (parmi 8). Au total, les différentes combinaisons entre les unités alpha et beta forment 24 dimères qui sont exprimés dans différents types cellulaires et qui se lient à différentes protéines de la matrice extra-cellulaire comme le collagène ou la fibronectine. 
Les intégrines relient la matrice extra-cellulaire au cytosquelette d'actine sous la membrane plasmique. Du côté interne de la membrane, l'intégrine se lie à des protéines comme la vinculine, la paxiline, la zyxine, ou la taline pour former des complexes appelés adhésions focales. 
Les intégrines sont la porte d'entrée des signaux mécaniques de la MEC. La plupart des molécules qui s'associent aux intégrines sont impliquées dans la mécanotransduction car elles sont responsables du déclenchement de voies de signalisation en réponse aux signaux reçus par les intégrines. 

\subsection{Cytosquelette}

Le cytosquelette est l'armature sur laquelle repose la cellule pour maintenir sa forme. Il lui permet d'exercer et de sentir des forces, de se déplacer, de gérer l'organisation interne de ses organites et le trafic entre elles. Il est composé de trois réseaux de protéines assemblées en filaments : les microtubules, les filaments intermédiaires et l'actine. 

Les filaments du cytosquelette sont en réorganisation constante : la polymérisation\footnote{Ici, il ne s'agit pas d'une polymérisation au sens chimique. La structure des filaments du cytosquelette est semblable à celle d'un polymère mais à une échelle différente, et les interactions chimiques en jeu sont tout à fait différentes. Par analogie, on parle de monomères, de polymérisation et de protéines de pontage.} et la dépolymérisation ont lieu en permanence, à des rythmes qui sont étroitement régulés par la cellule. 
Des protéines de pontage peuvent lier ces filaments pour former un gel réticulé, tandis que d'autres protéines peuvent se lier aux extrêmité pour contrôler la cinétique de polymérisation ou de dépolymérisation. Certaines des protéines de pontage sont des moteurs : elles peuvent convertir de l'énergie chimique en déplacement le long des filaments et mettre le gel de filaments sous tension. 

\subsubsection{Microtubules}

La tubuline, composée de deux sous-unité (alpha et bêta), s'associe en un protofilament de treize monomères, qui vont s'additionner pour former de longs filaments creux de diamètre 25nm. Ces filaments sont polarisés, une extrémité ne comportant que des sous-unités beta ( notée +) et l'autre que des alpha (notée - ), la polymérisation ayant lieu à l'extrémité + et la dépolymérisation à l'extrémité -. 
Le microtubule est le filament le plus rigide du cytosquelette, sa longueur de persistance est de l'ordre du millimètre, bien plus que la taille d'une cellule. 

Les microtubules sont organisés autour d'un élément central : le centrosome. A partir de cet élément central, les microtubules irradient vers la périphérie cellulaire. 
Sur les microtubules, deux moteurs moléculaires se déplacent, les dynéines et les kinésines, respectivement de l'extrémité + vers - et de - vers +.  

Les microtubules organisent les éléments de la cellule et assurent le transport entre les organites, en particulier par le déplacement des vésicules. Les moteurs moléculaires conduisent les vésicules d'un organite à l'autre, par exemple les protéines nouvellement synthétisées vers l'appareil de Golgi, ou les protéines de la matrice extra-cellulaire vers la membrane. 

Les microtubules jouent un rôle prépondérant lors de la mitose. Ils s'assemblent en un fuseau à la pointe duquel se trouve les centrosomes. Au centre s'alignent les paires de chromosomes, qui sont séparés et emmenés vers les centrosomes. La plupart des drogues perturbant les microtubules bloquent la mitose et sont utilisées pour cette raison dans les traitement anti-cancéreux. 

Les flagelles et les cils, par exemple le flagelle du spermatozoïde, sont composés d'un faisceau de microtubules animé par les moteurs moléculaires. 

\subsubsection{Filaments intermédiaires}

Contrairement à l'actine et à la tubuline, qui sont conservées et exprimées dans tous les types cellulaires, les filaments intermédiaires sont une famille de protéines dont l'expression dépend du type cellulaire (à l'exception de la lamine). 
Leur assemblage ne nécessite pas non plus l'hydrolyse d'ATP ou de GTP, il est spontané, et il n'existe pas de moteurs moléculaires se déplaçant sur ces filaments. 

Les réseaux de filaments intermédiaires sont liés aux protéines transmembranaires (cadhérines, intégrines) au niveau de structures appelées desmosomes (nom qui vient de la desmine) qui participent à l'intégrité mécanique des tissus. 

Le réseau de filaments intermédiaires est beaucoup plus stable que les deux autres réseaux du cytosquelette, qui sont en construction et destruction permanentes. Leur rôle est principalement d'ancrer les différents organites dans la cellule. 

Les plus connus des filaments intermédiaires sont sans doute les kératines, exprimées dans les cellules épithéliales et qui sont le constituant principal des poils et des ongles des mammifères (la kératine des reptiles et des oiseaux ne présente pas d'homologie avec celle des mammifères). 

La vimentine est exprimée dans toutes les cellules d'origine mésenchymateuse. Elle joue un rôle dans la localisation des vésicules bien que ne participant pas directement à leur transport : le réseau de vimentine interagit avec celui des microtubules. Elle est également impliquée dans la régulation de l'adhésion et de la migration cellulaire. 
L'expression de vimentine est souvent un marqueur de la transition épitéhlio-mésenchymateuse. 

Plusieurs types de filaments intermédiaires sont exprimés dans les neurones, et deux types sont spécifiquement responsables de la transparence du cristallin. 

\subsubsection{Les lamines}

Les lamines diffèrent des autres filaments intermédiaires sur plusieurs points. Elles ne forment pas un réseau dans toute la cellule mais sont localisées à la membrane nucléaire, et elles sont exprimées dans tous les types cellulaires. 

Les lamines forment un réseau soutenant la membrane nucléaire interne, dans lequel sont ancrés les pores nucléaires. Par l'intermédiaire de protéines comme la nesprine, le réseau laminaire est couplé mécaniquement au cytosquelette d'actine de la cellule et il est couplé au réseau interne par l'émerine, protéine qui coiffe la pointe des filaments d'actine et augmente leur polymérisation. 
Le réseau de lamines doit maintenir l'intégrité du noyau et définit ses propriétés mécaniques. 
Les lamines organisent également la chromatine à l'intérieur du noyau et contribuent à réguler l'expression du génome. 

La progéria et le syndrôme de dystrophie musculaire d'Emery-Dreifuss sont liés à des mutations sur le gène de la lamine A (ou sur celui de l'émerine dans ce deuxième cas). 


\subsubsection{Filaments intermédiaires dans les cellules musculaires}

Quatre types de filaments intermédiaires sont exprimés dans la cellule musculaire : la desmine, la vimentine, la lamine et la synésine. 
La desmine et la vimentine ont des structures suffisamment proches pour pouvoir former des hétéro-dimères et s'associer dans des filaments. La synésine ne peut former que des hétéro-dimères avec les autres filaments intermédiaires. Elle permet de lier le réseau de desmine au disque Z. 

Le réseau de desmine du muscle squelettique est principalement localisé au niveau des disques Z et organise leur alignement. Dans les desminopathies, les mutations de la desmine désorganisent les myofibrilles et causent une myopathie. 
Le réseau de desmine s'étend à travers toute la cellule musculaire et couple mécaniquement les différents organites. 
Les perturbations du réseau de desmine ne désorganisent pas le réseau de microtubules ni celui d'actine, mais il existe des liens entre les trois réseaux du cytosquelette. 

Dans les myoblastes (cellules précurseurs du muscle squelettique), la desmine participe aux propriétés mécaniques de la cellule : la surexpression de desmine sauvage augmente la rigidité cellulaire. 





\subsubsection{Actine}
  
L'actine constitue le réseau du cytosquelette le plus versatile et le plus dynamique. Son réseau est constamment réorganisé, et elle est le composant essentiel de la motilité cellulaire. 

Toutes les cellules eucaryotes expriment des actines, qui sont hautement conservées de la levure jusqu'à l'humain. Ce sont également des protéines très exprimées, et l'actine peut représenter jusqu'à 15\% de la masse de protéines dans une cellule. 

La membrane plasmique est une bicouche lipide qui n'a pas de résistance mécanique propre. Un réseau dense et très branché d'actine forme une couche rigide sous la membrane et confère sa forme à la cellule : le cortex d'actine. Ses propriétés sont essentielles dans les déformations des cellules et dans l'interaction de celles-ci avec un subtrat. 
Dans le volume de la cellule, l'actine forme un réseau moins dense, caractérisé par des faisceaux de filaments appelés fibres de stress. 
Dans le muscle, l'actine et ses moteurs associés (les myosines) forment une organisation spécialisée responsable de la contraction musculaire, appelée sarcomère. 

Lors de la division cellulaire, le réseau d'actine est dépolymérisé. À la fin de la séparation des chromosomes, l'actine corticale forme un anneau contractile qui va se refermer pour séparer le corps cellulaire en deux cellules filles distinctes. 

Un grand nombre de protéines interagissent avec l'actine (Actin-Binding Proteins). Concernant le cytosquelette d'actine, il s'agit de facteurs de polymérisation ou de dépolymérisation, de moteurs (les myosines) ou de protéines qui ancrent le réseau d'actine aux protéines transmembranaires (comme les cadhérines ou les intégrines) ou au réseau de lamines du noyau. 

L'actine n'est pas limitée à un rôle mécanique, elle a également des rôles de régulation de l'expression du génome et d'organisation de l'ADN. 

Le couplage entre le rôle mécanique de l'actine et son rôle transcriptionnel est au c\oe ur de ce travail de thèse, c'est pourquoi l'actine sera présentée en détail dans un chapitre dédié. 




\section{Expression du génome}

Le génome d'un organisme contient toute l'information nécessaire à la reconstitution de l'organisme entier. Cependant, il est évident dans un organisme pluricellulaire que si toutes les cellules contiennent le même génome, elles ne l'expriment pas de la même manière. C'est également le cas chez des être unicellulaires : des bactéries ou des levures ayant le même génome ne vont pas l'exprimer de la même manière selon les conditions extérieures. 

Les parties codant directement pour des protéines ne représentent qu'une toute petite partie de l'ADN. Autour des séquences codantes, le reste du génome permet de déterminer quelles protéines doivent être synthétisées, et dans quelles quantités. 

\subsection{Des gènes aux protéines}

Le chemin d'un gène à une protéine fonctionnelle passe par trois étapes principale : la transcription de l'ADN en ARN, la maturation de l'ARN et la traduction de l'ARN en protéine 

\subsubsection{La transcription}

L'information génétique est stockée dans l'ADN, pour y être protégée et transmise d'une génération à l'autre. Les quatre bases de l'ADN fonctionnent par paires, et grâce à ce mécanisme un brin d'ADN peut être complété par sa séquence miroir. 

Lorsque toutes les conditions d'expression sont réunies, l'ARN polymérase peut se fixer sur l'ADN au site de début de transcription. Le complexe ouvre l'ADN, et permet aux bases de l'ARN, Adénine, Uracile, Guanine et Cytosine, de compléter les bases de l'ADN, respectivement Thymosine, Adénine, Cytosine et Guanine. L'ARN polymérase progresse ainsi jusqu'à rencontrer une séquence terminateur sur l'ADN. 

À l'issu de cette étape, un ARN a été transcrit directement à partir de la séquence codante. 

\subsubsection{La maturation de l'ARN}

A l'issu de sa transcription, l'ARN subit plusieurs transformations. Des éléments sont ajoutés à ses deux extrémités, pour sa stabilité et sa reconnaissance par les ribosomes. 
L'ARN transcrit comprend deux types de séquences, les introns et les exons. Seuls les exons contiennent l'information des acides aminés pour coder la protéine. 
L'ARN va subir une étape d'épissage : les introns sont retirés de l'ARN jusqu'à ce qu'il ne reste que l'enchaînement des exons. Lors de cette étape, tous les exons peuvent être réassemblés, ou seulement une partie d'entre eux. Les protéines synthétisées par ces reconstitutions différentes seront différentes : on parle d'épissage alternatif. 
Un même gène peut donc coder pour des protéines différentes grâce au jeu de l'épissage. Chez la drosophile par exemple, un unique gène code pour autant de protéines différentes que tout le reste du génome. 

Après sa maturation, l'ARN messager est exporté du noyau pour se diriger vers le réticulum endoplasmique.

Tous les ARN n'ont pas vocation à être traduits en protéines : certains vont accomplir leur fonction sous cette forme. C'est le cas des éléments de la machine de traduction (ARN de transfert et ARN du ribosome), ou des micro-ARN, qui vont participer à la régulation de l'expression des gènes en aval de la transcription.

\subsubsection{La traduction}

Le code génétique est la correspondance entre des triplets de nucléotides et les acides aminés qui composent les protéines. 
Dans le cytoplasme, les ribosomes et les ARN de transfert vont faire la correspondance entre les nucléotides de l'ARN messager et les acides aminés, et assembler la suite des acides aminés jusqu'à atteindre un codon stop. 


\subsection{Régulation de l'expression du génome}

\subsubsection{Réorganisation de la chromatine}

L'ADN est stocké dans la cellule sous une forme condensée, la chromatine. Il est enroulé autour de complexes de protéines appelées les histones. Selon leur état biochimique (acétylation, méthylation) les histones forment un enroulement plus ou moins compact de l'ADN. 
Sous sa forme la plus condensée, l'hétérochromatine, l'ADN n'est pas accessible pour être lu par les ARN polymérase, il ne peut pas être transcrit. Par exemple, chez les femelles mammifères, l'un des deux chromosomes X est désactivé afin de ne pas avoir deux fois plus de transcription des gènes portés par ce chromosome qu'un organisme mâle. Il est condensé définitivement sous forme d'hétérochromatine. 

La compaction de l'ADN dans l'hétérochromatine ou dans l'euchromatine détermine donc quels gènes sont accessibles pour être trancrits et quels gènes sont désactivés dans l'hétérochromatine. 
La régulation de cette organisation par la modification chimique des histones est donc la première étape de la régulation trancriptionnelle du génome. 

\subsubsection{Les facteurs de transcription}

L'ARN polymérase ne peut pas se lier seule de manière stable sur l'ADN pour initier la transcription. 
Les facteurs de transcription sont une famille de protéines qui ont pour rôle de se fixer sur l'ADN en amont de la séquence à transcrire pour contribuer à l'expression du gène ou au contraire pour la réprimer. 

Un facteur de transcription reconnaît une séquence spécifique sur l'ADN en amont du gène régulé, le promoteur. Les promoteurs en amont d'un gène vont déterminer quels facteurs de transcription vont être capables d'activer la transcription. Des gènes partageant le même promoteur vont être activés par les mêmes facteurs de transcription et répondre aux mêmes stimuli.  

Dans le développement des êtres pluricellulaires, la différenciation des cellules souches totipotentes de l'embryon en différents types de tissus va être orchestrée par l'activation de nombreux facteurs de transcription. 

Les facteurs de transcription vont également être impliqués dans les réponses d'une cellule ou d'un organisme à des signaux biologiques provenant de cellules voisines (par l'intermédiaire des liaisons transmembranaires entre deux cellules, par une structure spécialisée comme un synapse ou par un signal paracrine) ou des tissus éloignés (hormones circulant dans le corps), mais aussi à des signaux environnementaux comme la température, le choc osmotique, les contraintes mécaniques, l'exposition à la lumière du soleil  \dots


Un facteur de transcription peut recruter d'autres protéines, comme des coactivateurs (ou des corépresseurs), ou des protéines qui vont changer localement l'état de compacité de la chromatine, afin de rendre le gène plus facilement ou plus difficilement accessible. 

Un facteur de transcription ou ses coactivateurs doivent être présents dans le noyau pour accomplir leur fonction. Le contrôle de leur localisation dans la cellule permet d'activer ou de désactiver un facteur de transcription. Par exemple, le récepteur des \oe strogènes est principalement présent dans le cytoplasme en l'absence d'hormone. En présence d'hormones, il est transporté dans le noyau où il peut se lier à l'ADN et activer la trancription de ses gènes cibles. 


Un facteur de transcription peut également être régulé par la phosphrylation ou celle de ses cofacteurs, ou par la présence d'un ligand. 


\end{document}