 \chapter{La cellule musculaire}
 
 Les muscles sont les éléments de l'organisme dont la fonction mécanique est la plus évidente : le rôle principal des muscles est d'exercer et de sentir des forces (un de leur rôles secondaires est de produire de la chaleur pour maintenir notre température corporelle). 
 Nous possédons trois types de muscles : le muscle cardiaque, les muscles lisses comme ceux des intestins ou l'utérus, et les muscles squelettiques comme le biceps ou le diaphragme.
 
 Les muscles, en réponse à des signaux biochimiques, sont capables de se contracter pour produire des forces, nécessaires à la circulation sanguine, au transit dans l'appareil digestif ou aux mouvements volontaires des membres. Seuls les muscles squelettiques peuvent être contrôlés volontairement : on peut ainsi contrôler sa respiration, mais pas ses battements cardiaques. 
 
 
 Les muscles sont un tissu particulièrement plastique, ils vont s'adapter à la charge de travail qui leur est demandée. 
 Ainsi, nous avons tous l'expérience de la musculation : un surcroît de stimulation des muscles va provoquer leur développement. À l'inverse, une immobilisation forcée, suite à une blessure par exemple, la masse musculaire diminue rapidement. 
 La fibre musculaire est donc un lieu privilégié de la mécanotransduction : un signal mécanique est transmis au muscle par l'intermédiaire des contractions qu'on lui demande d'effectuer, et celui-ci y répond pas des conséquences biologiques, la synthèse de plus de masse musculaire. 
 

 
 
\section{Les fibres musculaires}


 \section{Du myoblaste à la fibre musculaire}
 