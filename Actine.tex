\documentclass{report}
\usepackage[T1]{fontenc}
\usepackage[utf8]{inputenc}
\usepackage[francais]{babel}
\usepackage{amsmath}
\usepackage{graphicx}
\usepackage[backend=biber,style=authoryear,bibencoding=utf8]{biblatex}
\usepackage[colorlinks,linkcolor=blue]{hyperref}

\begin{document}
\chapter{L'actine}

L'actine est une protéine ubiquitaire conservée chez tous les eucaryotes, exprimée dans tous les types cellulaires. 
Ses fonctions sont multiples et variées et se divisent en deux catégories principales, les fonctions mécaniques et les fonctions régulatrices. 

Elle se présente dans la cellule sous deux formes principales : en monomères (Actine G pour globulaire) ou en filaments (Actine F). 
Elle interagit avec un grand nombre de protéines (certains pensent même qu'il s'agit de la protéine interagissant avec le plus grand nombre d'autres protéines) appelées Actin-Binding Proteins. 

L'actine est un composant du cytosquelette sous forme d'un réseau de filaments très dynamique. La rigidité d'une cellule et sa motilité sont majoritairement contrôlées par l'organisation du cytosquelette d'actine. 

Mais l'actine est également un composant des trois ARN Polymérases PolI, PolII et PolIII, qui transcrivent l'ADN en ARN pendant la première étape de l'expression du génome. 
Elle est indispensable à la réorganisation de la chromatine qui précède l'expression mais aussi à l'export de l'ARN.  

L'association de ces rôles mécaniques et biologiques fait de l'actine un acteur de choix dans l'interface entre les signaux mécaniques et les signaux biologiques. 

Dans le corps, l'actine a des fonctions spécifiques dans un grand nombre d'organes, comme la contraction des muscles, l'organisation des dendrites et des axones des neurones, le fonctionnement des plaquettes ou de l'appareil auditif. 



\section{Actine G}

Chez les mammifères, l'actine est codée par 6 gènes qui peuvent donner une trentaine de molécules différentes par le jeu de l'épissage. 
Elles sont divisées en trois familles : les actines $\alpha$ qui sont exprimées dans les muscles cardiaques, lisses et squelettiques, les actines $\beta$ et $\gamma$ exprimées dans les autres types cellulaires.
Les différentes formes d'actine sont très proches en séquence, mais ne peuvent pas complètement se substituer les unes aux autres. Toutes les formes peuvent s'incorporer dans les filaments. 

La protéine transcrite a un poids moléculaire de 42 kDa, et est produite en grande quantité dans les cellules, où elle pèse environ pour 1 à 5 \% de la masse protéique. 

\subsection{Structure}
La structure moléculaire de l'actine a été observée un grand nombre de fois, en cristallisation avec différents ABP comme la Dnase, la latrunculine ou la profiline. 


Elle est composée de 4 sous-domaines organisés en deux lobes. Les sous-domaines 2 et 4 forment l'extrémité - du filament, les sous-domaines 1 et 3 forment l'extrémité +. Entre les deux lobes se trouve le site de liaison à l'ATP. 
À côté de ce site se trouve une zone d'interaction avec les cations divalents (Ca$^{2+}$ ou Mg$^{2+}$). 

Dans le sous-domaine 2 se trouve une structure de 8 acides aminés appelée Dnase binding loop, désorganisée dans la plupart des cristallisations de l'actine mais organisée en feuillet $\beta$ lorsqu'elle est liée à la Dnase. Au centre de cet élément se trouve une méthionine en position 44 qui peut être oxydée par la protéine MICAL. 

\subsection{L'actine est une ATPase}

L'actine, après sa fabrication, n'est prête à jouer son rôle qu'avec l'ajout d'une ATP au centre de sa structure.
En plus de ses deux formes, globulaire ou filamenteuse, l'actine a donc également deux états énergétiques : ATP ou ADP.

Les deux formes d'actine, ATP et ADP sont capables de former des filaments et de s'incorporer à des filaments. 
Cependant, l'actine ATP est plus facilement polymérisée alors que l'actine ADP est plus facilement dépolymérisée. 
Les actines ATP incorporés dans un filament sont ensuite hydrolysées, et deviennent donc plus facilement dépolymérisables, ce qui donne lieu à un tapis roulant : les monomères s'ajoutent à un bout et s'enlèvent à l'autre. 

\subsection{Localisation et transport}

L'actine a longtemps été étudiée pour son rôle dans le cytoplasme, en tant que composant du cytosquelette. Cependant de l'actine est également présente dans le noyau de la cellule, où elle a des rôles essentiels. 
L'actine peut polymériser dans les deux compartiments, bien que l'on ne trouve pas de grands filaments organisés dans le noyau. 

L'actine est à une taille intermédiaire pour les pores nucléaires : elle n'est pas tout à fait assez petite pour diffuser facilement à travers. Elle est donc transportée activement entre le noyau et le cytoplasme. 
Son import nécessite la liaison à la cofiline, et est médiée par l'importine 9. Son export nécessite la profiline et est médiée par l'exportine 6. 



\subsection{Oxydation de l'actine}

Les protéines MICAL sont capables d'ajouter deux atomes d'oxygène sur la Met44 de l'actine. Cet acide aminé est au centre d'une zone de la protéine qui relie les monomères dans un filament. 
L'oxydation spécifique de cet acide aminé déstabilise les filaments et l'actine oxydée est incapable de se lier à d'autres actines pour former de nouveaux filaments. 

MICAL2 est une version de cette protéine localisée dans le noyau. L'actine qu'elle oxyde, en plus d'être dépolymérisée, est expulsée du noyau et ne peut plus y entrer. MICAL2 organise donc la régulation de l'actine nucléaire en appauvrissant le réservoir d'actine nucléaire en filaments et en monomères. 




\section{Actine F}

La principale fonction de l'actine chez les eucaryote est sa capacité à former un réseau de filaments branchés, connecté par des moteurs moléculaires. 
La formation de filaments d'actine est régulée par de très nombreuses protéines qui vont se lier aux monomères ou aux filaments. 

\subsection{Le filament}

Les filaments d'actine sont très dynamiques, et ont une structure changeante. Ils ont un diamètre de 6nm et une longueur de persistance de l'ordre de la dizaine de microns, donc du même ordre de grandeur que la taille typique cellulaire. 

Les monomères d'actine s'associent les unes à la suite des autres, l'extrémité pointue d'un monomère se liant à l'extrémité barbée de l'autre, avec une rotation entre un monomère et l'autre. 
Le filament est alors polarisé, avec une extrémité pointue notée également - et une extrémité barbée notée +. 

Les deux bouts du filaments ont des affinités différentes pour les monomères. En présence d'une grande quantité de monomères disponibles, le filament peut croître par les deux bouts, mais dans une concentration intermédiaire, le filament va croître par ajout de monomères à son extrémité + et décroître par dépolymérisation à l'extrémité -. L'équilibre entre les deux cinétiques de réaction détermine si la taille du filament croît ou non. 

Cet équilibre est connu comme le "tapis roulant" de l'actine : lorsque les deux cinétiques sont égales, le filament avance par remplacement des monomères en gardant une longueur constante. 

Bien que cet assemblage puisse avoir lieu spontanément en présence d'actine, de nombreuses protéines aident à la nucléation des filaments, à leur stabilité ou à leur désabilisation. 

\subsection{L'équilibre de polymérisation}

Les dimères et les trimères d'actine sont des structures peu stables, à la durée de vie assez courte. C'est à partir du tétramère que la structure devient suffisamment stable pour créer un nouveau filament d'actine. 

Afin de dépasser cette barrière, d'autres protéines jouent le rôle de nucléateurs. Le complexe Arp2/3 (Arp pour Actin Related Protein) est le plus connu de ces nucléateurs. Il se lie au côté d'un filament et Arp2 et Arp3 miment un dimère d'actine. D'autres monomères peuvent alors se fixer sur cette base et un nouveau filament peut croître. Ce nouveau filament est de plus attaché avec un angle fixé au filament initial, créant un réseau branché. 

Les formines se fixent à l'extrémité barbée d'un filament et y ajoutent successivement des monomères d'actine. Les formines peuvent nucléer un nouveau filament en stabilisant un dimère et en y ajoutant d'autres monomères. 


Une fois les filaments formés, des facteurs d'élongation comme les formines ou Ena/VASP, peuvent ajouter des monomères liés à la profiline à leur extrémité barbée.
Des protéines de coiffage se lient à une extrémité du filament et empêchent l'ajout de nouveaux monomères, comme CapZ à l'extrémité +  et la troponine à l'extrémité -. Elles stabilisent les filaments existant dans les structures très organisées des muscles. 

La profiline se lie aux monomères d'actine, mais son action sur l'équilibre des filaments est complexe.  

\subsection{Organisation en réseau de filaments}


\section{Rôle mécanique de l'actine}

\section{Rôle régulateur de l'actine}




\end{document}