\documentclass{report}
\usepackage[T1]{fontenc}
\usepackage[utf8]{inputenc}
\usepackage[francais]{babel}
\usepackage{amsmath}
\usepackage{graphicx}
\usepackage[backend=biber,style=authoryear,bibencoding=utf8]{biblatex}
\usepackage[colorlinks,linkcolor=blue]{hyperref}

\begin{document}
\chapter{L'actine}

L'actine est une protéine ubiquitaire conservée chez tous les eucaryotes, exprimée dans tous les types cellulaires. 
Ses fonctions sont multiples et variées et se divisent en deux catégories principales, les fonctions mécaniques et les fonctions régulatrices. 

Elle se présente dans la cellule sous deux formes principales : en monomères (Actine G pour globulaire) ou en filaments (Actine F). 
Elle interagit avec un grand nombre de protéines (certains pensent même qu'il s'agit de la protéine interagissant avec le plus grand nombre d'autres protéines) appelées Actin-Binding Proteins. 

L'actine est un composant du cytosquelette sous forme d'un réseau de filaments très dynamique. La rigidité d'une cellule et sa motilité sont majoritairement contrôlées par l'organisation du cytosquelette d'actine. 

Mais l'actine est également un composant des trois ARN Polymérases PolI, PolII et PolIII, qui transcrivent l'ADN en ARN pendant la première étape de l'expression du génome. 
Elle est indispensable à la réorganisation de la chromatine qui précède l'expression mais aussi à l'export de l'ARN.  

L'association de ces rôles mécaniques et biologiques fait de l'actine un acteur de choix dans l'interface entre les signaux mécaniques et les signaux biologiques. 

Dans le corps, l'actine a des fonctions spécifiques dans un grand nombre d'organes, comme la contraction des muscles, l'organisation des dendrites et des axones des neurones, le fonctionnement des plaquettes ou de l'appareil auditif. 



\section{Actine G}

Chez les mammifères, l'actine est codée par 6 gènes qui peuvent donner une trentaine de molécules différentes par le jeu de l'épissage. 
Elles sont divisées en trois familles : les actines $\alpha$ qui sont exprimées dans les muscles cardiaques, lisses et squelettiques, les actines $\beta$ et $\gamma$ exprimées dans les autres types cellulaires.
Les différentes formes d'actine sont très proches en séquence, mais ne peuvent pas complètement se substituer les unes aux autres. Toutes les formes peuvent s'incorporer dans les filaments. 

La protéine transcrite a un poids moléculaire de 42 kDa, et est produite en grande quantité dans les cellules, où elle pèse environ pour 1 à 5 \% de la masse protéique. 

\subsection{Structure}
La structure moléculaire de l'actine a été observée un grand nombre de fois, en cristallisation avec différents ABP comme la Dnase, la latrunculine ou la profiline. 


Elle est composée de 4 sous-domaines organisés en deux lobes. Les sous-domaines 2 et 4 forment l'extrémité - du filament, les sous-domaines 1 et 3 forment l'extrémité +. Entre les deux lobes se trouve le site de liaison à l'ATP. 
À côté de ce site se trouve une zone d'interaction avec les cations divalents (Ca$^{2+}$ ou Mg$^{2+}$). 

Dans le sous-domaine 2 se trouve une structure de 8 acides aminés appelée Dnase binding loop, désorganisée dans la plupart des cristallisations de l'actine mais organisée en feuillet $\beta$ lorsqu'elle est liée à la Dnase. Au centre de cet élément se trouve une méthionine en position 44 qui peut être oxydée par la protéine MICAL. 

\subsection{L'actine est une ATPase}

L'actine, après sa fabrication, n'est prête à jouer son rôle qu'avec l'ajout d'une ATP au centre de sa structure.
En plus de ses deux formes, globulaire ou filamenteuse, l'actine a donc également deux états énergétiques : ATP ou ADP.

Les deux formes d'actine, ATP et ADP sont capables de former des filaments et de s'incorporer à des filaments. 
Cependant, l'actine ATP est plus facilement polymérisée alors que l'actine ADP est plus facilement dépolymérisée. 
Les actines ATP incorporés dans un filament sont ensuite hydrolysées, et deviennent donc plus facilement dépolymérisables, ce qui donne lieu à un tapis roulant : les monomères s'ajoutent à un bout et s'enlèvent à l'autre. 

\subsection{Localisation et transport}

L'actine a longtemps été étudiée pour son rôle dans le cytoplasme, en tant que composant du cytosquelette. Cependant de l'actine est également présente dans le noyau de la cellule, où elle a des rôles essentiels. 
L'actine peut polymériser dans les deux compartiments, bien que l'on ne trouve pas de grands filaments organisés dans le noyau. 

L'actine est à une taille intermédiaire pour les pores nucléaires : elle n'est pas tout à fait assez petite pour diffuser facilement à travers. Elle est donc transportée activement entre le noyau et le cytoplasme. 
Son import nécessite la liaison à la cofiline, et est médiée par l'importine 9. Son export nécessite la profiline et est médiée par l'exportine 6. 



\subsection{Oxydation de l'actine}

Les protéines MICAL sont capables d'ajouter deux atomes d'oxygène sur la Met44 de l'actine. Cet acide aminé est au centre d'une zone de la protéine qui relie les monomères dans un filament. 
L'oxydation spécifique de cet acide aminé déstabilise les filaments et l'actine oxydée est incapable de se lier à d'autres actines pour former de nouveaux filaments. 

MICAL2 est une version de cette protéine localisée dans le noyau. L'actine qu'elle oxyde, en plus d'être dépolymérisée, est expulsée du noyau et ne peut plus y entrer. MICAL2 organise donc la régulation de l'actine nucléaire en appauvrissant le réservoir d'actine nucléaire en filaments et en monomères. 




\section{Actine F}

La principale fonction de l'actine chez les eucaryote est sa capacité à former un réseau de filaments branchés, connecté par des moteurs moléculaires. 
La formation de filaments d'actine est régulée par de très nombreuses protéines qui vont se lier aux monomères ou aux filaments. 

\subsection{Le filament}

Les filaments d'actine sont très dynamiques, et ont une structure changeante. Ils ont un diamètre de 6nm et une longueur de persistance de l'ordre de la dizaine de microns, donc du même ordre de grandeur que la taille typique cellulaire. 

Les monomères d'actine s'associent les unes à la suite des autres, l'extrémité pointue d'un monomère se liant à l'extrémité barbée de l'autre, avec une rotation entre un monomère et l'autre. 
Le filament est alors polarisé, avec une extrémité pointue notée également - et une extrémité barbée notée +. 

Les deux bouts du filaments ont des affinités différentes pour les monomères. En présence d'une grande quantité de monomères disponibles, le filament peut croître par les deux bouts, mais dans une concentration intermédiaire, le filament va croître par ajout de monomères à son extrémité + et décroître par dépolymérisation à l'extrémité -. L'équilibre entre les deux cinétiques de réaction détermine si la taille du filament croît ou non. 

Cet équilibre est connu comme le "tapis roulant" de l'actine : lorsque les deux cinétiques sont égales, le filament avance par remplacement des monomères en gardant une longueur constante. 

Bien que cet assemblage puisse avoir lieu spontanément en présence d'actine, de nombreuses protéines aident à la nucléation des filaments, à leur stabilité ou à leur déstabilisation. 

\subsection{L'équilibre de polymérisation}

Les dimères et les trimères d'actine sont des structures peu stables, à la durée de vie assez courte. C'est à partir du tétramère que la structure devient suffisamment stable pour créer un nouveau filament d'actine. 

Afin de dépasser cette barrière, d'autres protéines jouent le rôle de nucléateurs. Le complexe Arp2/3 (Arp pour Actin Related Protein) est le plus connu de ces nucléateurs. Il se lie au côté d'un filament et Arp2 et Arp3 miment un dimère d'actine. D'autres monomères peuvent alors se fixer sur cette base et un nouveau filament peut croître. Ce nouveau filament est de plus attaché avec un angle fixé au filament initial, créant un réseau branché. 

Les formines se fixent à l'extrémité barbée d'un filament et y ajoutent successivement des monomères d'actine. Les formines peuvent nucléer un nouveau filament en stabilisant un dimère et en y ajoutant d'autres monomères. 


Une fois les filaments formés, des facteurs d'élongation comme les formines ou Ena/VASP, peuvent ajouter des monomères liés à la profiline à leur extrémité barbée.
Des protéines de coiffage se lient à une extrémité du filament et empêchent l'ajout de nouveaux monomères, comme CapZ à l'extrémité +  et la troponine à l'extrémité -. Elles stabilisent les filaments existant dans les structures très organisées des muscles. 

La profiline se lie aux monomères d'actine, mais son action sur l'équilibre des filaments est complexe. Son action empêche la formation de nouveau filaments, mais l'actine liée à la profiline est préférentiellement recrutée par les formines dans leur travail d'élongation. De plus elle catalyse le remplacement de l'ADP par une ATP dans l'actine qu'elle lie. Selon les conditions, la profiline peut donc être un élément qui promeut un réseau d'actine plus polmérisé, aux fibres plus longues. 

La cofiline est souvent considérée comme contrant la profiline, et sont action sur l'équilibre entre actine F et actine G est également complexe. La cofiline se lie aux filaments d'actine ADP et entraîne une configuration où la rotation des monomères les uns par rapport aux autres est plus grande. Cela déstabilise les filaments et les casse. 
Cependant, cette action donne naissance à de nouvelles extrémités + et donc permet d'initier l'élongation de nouveaux filaments. 
Selon les conditions, en particulier l'activation de facteurs d'élongation et la disponibilité des monomères, l'activation de la cofiline peut donc agir en faveur de la polymérisation ou son contraire. 



\subsection{Organisation en réseau de filaments}

Une fois les filaments formés, ils s'associent entre eux par l'intermédiaire de protéines de pontage qui peuvent être statiques ou mobiles. 

Les microfilaments peuvent être liés parallèlement pour former de longues fibres épaisses par des protéines comme la fimbrine. Ils peuvent aussi être liés en faisceaux anti-parallèles, par exemple par l'$\alpha$-actinine. 
Arp2/3 permet de nucléer de nouveaux filaments d'actine à partir du côté d'un filament déjà présent, avec un angle fixé de 70$\deg$. La filamine permet également de créer un maillage du réseau entre des filaments existants. 

\subsubsection{Les myosines}

Les myosines sont des moteurs moléculaires qui se déplacent sur l'actine en consommant de l'ATP. Il en existe chez tous les eucaryotes, mais leur homologie n'est pas aussi grande que celle de l'actine, car elles ont des fonctions différentes. Dans le génome humain, on dénombre une quarantaine de gènes pour la myosine. 

La myosine II, aussi appelée "conventionnelle" est la plus étudiée. Elle est présente en quantité importante dans le muscle, car avec l'actine elle permet la contraction musculaire. 

Les myosines ont une tête qui peut se lier à l'actine en filament, un cou  qui sert de levier et de régulateur, et une queue qui sert souvent à former un dimère, et éventuellement à se lier à un cargo. On les appelles "chaînes lourdes" par opposition aux "chaînes légères", qui ne sont pas à proprement parler des myosines mais qui sont des protéines qui vont se lier au cou des "chaînes lourdes" pour les réguler. 

Par exemple, pour la contraction musculaire, deux chaînes lourdes de myosine II s'associent en dimère par leur queue, et quatre chaînes légères s'ajoutent au niveau des deux cous. Le dimère a alors deux têtes pouvant se lier à l'actine, et va s'en servir comme de deux jambes pour avancer le long du filament. 
La myosine est une ATPase, lors de l'hydrolyse de l'ATP qui lui est attachée, sa tête va changer de conformation et se détacher de l'actine. L'ADP sera alors libérée, et remplacée par une nouvelle ATP.  La tête reprend alors sa conformation initiale, et peu se rattacher au filament. Dans le dimère, chaque tête va faire ainsi un pas successivement et faire avancer le moteur sur le filament en consommant de l'ATP. 



Certaines myosines ont un rôle analogue à celui des moteurs moléculaires associés aux microtubules et transportent des molécules le long des filaments, en général en direction de l'extrémité + (seule la myosine VI se déplace en sens inverse). 


Les faiceaux anti-parallèles peuvent être liés par des paires de dimères, qui vont marcher en sens opposé sur les deux filaments, et donc les déplacer l'un par rapport à l'autre. Si les deux filaments sont liés par ailleurs dans le réseau, il va être mis sous tension par ces moteurs moléculaires.




\section{Rôle mécanique de l'actine : du filament au cytosquelette}

Le cytosquelette est une structure multi-échelle, allant de l'échelle des moteurs moléculaires et des nucléateurs, à l'échelle de la cellule toute entière, en passant pas l'échelle des filaments et des réseaux de filaments. Il peut ressentir et exercer des forces à toutes les échelles. 


Dans cette partie, il ne s'agit pas de parler des propriétés mécaniques du cytosol ou de la cellule, mais d'expliquer comment les filaments peuvent générer des forces et comment il réagissent à des forces. 

\subsection{Mécanique du filament d'actine}

Le filament d'actine en lui-même est à la fois générateur et senseur de forces. 

 \subsubsection{Longueur de persistance}
La longueur de persistance est un moyen de quantifier la corrélation entre l'orientation des différents segments d'un polymère soumis aux fluctuations thermiques. Si l'on considère un polymère de longueur $L$ auquel on attribue une abscisse curviligne $s$, avec $\vec{t_s}$ la tangente au polymère en $s$, alors on a la relation : 
$$ \langle \vec{t_0}\cdot \vec{t_s} \rangle_L \propto e^{L/\ell_p}$$

Au bout de quelques longueurs de persistance, l'information de l'orientation du polymère en $s=0$ est perdue. 

La longueur de persistance dépend de l'énergie thermique disponible pour agiter le filament. Une manière de définir la rigidité d'un polymère indépendamment de la température consiste à définir un module de courbure comme le produit de la longueur de persistance et de l'énergie thermique : $$\kappa = \ell_p k_B T$$


Le filament d'actine subit un vieillissement par l'hydrolyse de l'ATP des monomères qui le composent. Le changement de conformation induit par l'hydrolyse de l'actine a des conséquences sur les propriétés mécaniques du filament. 
Un filament d'actine ATP a une longueur de persistance de 15 micromètres, contre 9 micromètres pour un filament d'actine ADP. Le vieillissement du filament le rend donc plus déformable et plus souple. 
Les protéines attachées au filament peuvent également, en stabilisant une conformation, changer sa rigidité. Un filament stabilisé par la phalloïdine ou par la tropomyosine voit sa longueur de persistance augmentée à 18 et 20 microns respectivement. Au contraire, la conformation stabilisée par la cofiline n'a qu'une longueur de persistance de 2,2 microns. 

Un même filament d'actine peut évidemment être le lieu de toutes ces modifications en même temps. Souvent l'extrêmité + des filaments est riche en actine ATP alors que l'extrémité - est riche en actine ADP, ce qui crée un filament plus rigide d'un côté et plus flexible de l'autre. 

\subsubsection{Couplage traction-torsion}

L'hélice que forme le filament peut adopter des conformations différentes, en particulier en ce qui concerne l'angle de rotation entre les monomères successifs. 
Une force tirant sur le filament va alors favoriser une conformation à faible torsion, ce qui va rendre la fixation de la cofiline, qui stabilise le filament dans une conformation à grande rotation, beaucoup plus difficile. 
Il en résulte que la cofiline est moins efficace sur les filaments qui sont en tension, induisant une préservation automatique des filaments sous contrainte par rapport aux filaments libres. 
Au contraire, mDia1 et la profiline sont plus efficaces sur les filaments soumis à une tension. 
La conformation de l'actine est alors un senseur de contrainte qui va encourager la préservation et l'élongation des filaments qui ressentent une force de traction. 

Les filaments d'actine semi-flexibles peuvent également être courbés, en particulier au voisinage de la membrane. 
À cause de l'organisation hélicoïdale des monomères, la courbure d'un filament d'actine exerce également une torsion sur ce filament, ce qui rend la modélisation des filaments encore plus complexe. 
Le facteur de nucléation Arp2/3 se lie plus facilement au côté convexe d'un filament d'actine courbé. 

\subsubsection{Les myosines}

\subsection{Mécanique des adhésions focales}

Les sites d'ancrage de la cellule dans la matrice extra-cellulaire sont la porte d'entrée des signaux mécaniques dans la cellule. Parmi les molécules qui constituent les adhésions focales, certaines réagissent directement lorsqu'elles sont soumises à des stimulations mécaniques. 

Lorsque la transmission des forces est coupée dans la cellule (par l'ajout de drogues qui inhibe la contractilité du cytosquelette), les adhésions focales disparaissent : la tension est nécessaire non seulement à leur constitution, mais également à leur maintien. Ces contraintes peuvent provenir de forces extérieures mais aussi de la contraction du cytosquelette sous l'action des moteurs moléculaires. 

En présence d'une force, les intégrines forment des agrégats et leur affinité pour le ligand augmente grâce à un changement de conformation. 
La taline relie les intégrines aux filaments d'actine. Dans la conformation initiale, elle est repliée sur elle-même. Lorsqu'elle est mise sous tension entre les intégrines et l'actine, elle se déplie, laissant apparaître des domaines de liaisons à la vinculine qui n'étaient pas précédemment accessibles. 
La vinculine peut alors recruter d'autres protéines dans les adhésions focales, comme l'$\alpha$-actinine, qui connecte les filaments d'actine en faisceaux. L'étirement de l'$\alpha$-actinine change sa configuration et pourrait être à l'origine de la relocalisation de la zyxine en réponse aux contraintes mécaniques. 

La filamine, qui lie entre eux des filaments d'actine, peut être dépliée lors de tensions sur les filaments. Son domaine de liaison aux intégrines devient alors accessible, et la filamine ancre alors le cytosquelette à la matrice extra-cellulaire par l'intermédiaire des intégrines. De plus, cela cause un changement de conformation des intégrines qui favorise la formation d'agrégats, renforçant l'adhésion. 

Ce ne sont là que quelques exemples de protéines impliquées dans les adhésions focales, il en existe des centaines. Leur exemple montre qu'au premier niveau de contact avec l'environnement mécanique extérieur, les forces sont transcrites en signal biologique au niveau de la molécule individuelle, en changeant la conformation des protéines. 



\subsection{Mécanique du cytosquelette d'actine}
cortex, podosomes, filopodes, lamellipode, fibres de stress, motilité. 
\subsection{Un cas particulier : la contraction musculaire}
myfibrilles, sarcomères, acto-myosine

\section{Rôle régulateur de l'actine}




\end{document}