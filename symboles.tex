\chapter*{Liste des symboles}

\noindent $A$ : préfacteur de la loi de puissance ajustant la fonction de fluage\\
$a$ : rayon de la bille\\
$\frac{\Delta A}{A}$ Variation d'aire imposée par l'étireur\\
$\alpha$ : exposant de la loi de puissance ajustant la fonction de fluage\\
$\vec{B}$ : champ magnétique\\
$d$ : diamètre de la bille\\
$E_c$ : énergie de courbure\\
$E$ : module d'Young\\
$\epsilon$ : déformation relative\\
$\dot{epsilon}$ : taux de déformation \\
$\eta$ : viscosité\\
$\dfrac{\eta}{T}$ : variation de la viscosité en fonction de la température\\
$G$ : module élastique ($G'$ sa partie réelle et $G''$ sa partie imaginaire)\\
$\gamma$ : conductivité électrique du cuivre\\
$\Gamma$ : fonction gamma \\
$h$ : distance entre la bille observée et les parois de la cuve\\
$\vec{H} $ : champ d'excitation magnétique \\
$h_u$ : épaisseur de la cellule entre la bille et la lamelle\\
$I,i$ : intensité du courant électrique\\
$J$ : fonction de fluage\\
$\kappa$ : module élastique de courbure\\
$k_B$ : constance de Boltzmann\\
$L$ : longueur\\
$l_b$ : longueur de la bobine\\
$l_p$ : longueur de persistance\\
$\vec{m}$ : moment magnétique induit dans la bille\\
$\mu_0$ : perméabilité magnétique du vide\\
$n$ : nombre de spire par mètre de la bobine\\
$n_c$ : nombre de couches de fil de la bobine\\
$\nu$ : coefficient de Poisson\\
$\omega$ : fréquence de stimulation\\ 
$p$ : facteur géométrique de la fonction de fluage\\
$R$ : résistance de la bobine\\
$\delta R(t)$ : distance entre la position de la bille à l'instant $t$ et sa position à $t=0$\\
$r_b$ : rayon de la bobine\\
$R_e$ : nombre de Reynold\\
$r_f$ : rayon du fil de cuivre\\
$R_p$ Rayon du plot de l'étireur\\
$\rho$ : masse volumique du fluide\\
$s$ : section du fil de cuivre de la bobine ou abscisse curviligne le long d'un filament d'actine\\
$\sigma$ : contrainte\\
$T$ : température\\
$\vec{t}$ : vecteur tangent à l'abscisse curviligne\\
$\theta$ : demi-angle d'enfoncement de la bille dans la cellule\\
$\tau$ : temps de relaxation \\
$\vec{U}$ : vitesse du fluide\\
$V$ : volume de la bille\\
$x$ : coordonnée de la bille selon l'axe $X$ d'application de la force\\
