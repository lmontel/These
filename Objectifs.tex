%\documentclass{report}
%\usepackage[T1]{fontenc}
%\usepackage[utf8]{inputenc}
%\usepackage[francais]{babel}
%\usepackage{amsmath}
%\usepackage{graphicx}
%\graphicspath{{Figures/}}
%
%\begin{document}
\chapter*{Objectifs}

Lors de l'introduction, une brève description des objectifs de ce travail avait été donnée. Après ces 5 chapitres introductifs qui décrivent le système que nous nous sommes proposées d'étudier, je vais pouvoir revenir avec bien plus de précision sur ces objectifs. 

Le système MRTF/SRF a principalement été étudié lors de son activation par des stimulations biochimiques, comme le sérum ou TGF-$\beta$, mais on sait que les signaux mécaniques peuvent également l'activer. 
En effet, la localisation de MRTF dans la cellule est régulée par l'actine, un des constituants principaux du cytosquelette. 
Tout signal qui vient perturber l'équilibre entre les filaments et les monomères d'actine a donc le potentiel d'influer sur la localisation de MRTF dans la cellule et donc sur l'activation de SRF. 
Deux voies principales et complémentaires ont été identifiées : la voie RhoA augmente la polymérisation de l'actine en favorisant l'élongation des filaments et en défavorisant leur destruction, ce qui encourage l'import de MRTF vers le noyau, tandis que la voie MICAL expulse l'actine hors du noyau, ce qui bloque la sortie du noyau. Ces deux voies concourent à une MRTF-A localisée dans le noyau, où elle peut activer la transcription par SRF. 

L'objectif principal de cette thèse est alors d'étudier plus en détail l'activation mécanique de MRTF/SRF par l'intermédiaire de l'actine. Comment l'actine se réorganise-t-elle suite à une stimulation mécanique, et comment cela influe-t-il sur la localisation de MRTF-A dans la cellule ? La réorganisation du cytosquelette d'actine en réponse à une stimulation mécanique est-elle suffisante pour accumuler MRTF-A dans le noyau ? Comment le système répond-il à des contraintes mécaniques d'intensité différentes, à une échelle locale ou globale ? Quelles sont les échelles de temps mises en jeu ? Quelles voies sont activées ? Les propriétés mécaniques des cellules changent-elles ? 

Pour répondre à ces questions, il faut coupler les techniques de physique et de biologie : construire des dispositifs permettant d'exercer des contraintes contrôlées à l'échelle de la cellule et en même temps de visualiser en microscopie de fluorescence les protéines en jeu, actine et MRTF-A. 
Pour cela j'ai conçu des pinces magnétiques, qui permettent d'exercer des forces locales et contrôlées au niveau d'une cellule unique, et un étireur de cellules qui permet d'appliquer une déformation statique au substrat sur lesquelles sont les cellules. 
Les deux techniques permettent l'observation en direct en microscopie de fluorescence de la localisation des protéines. 
MRTF-A et l'actine ont été visualisées en immunofluorescence, ou à l'aide de versions fluorescentes transfectées dans les cellules (MRTF-A GFP et actine mCherry). L'observation de l'actine s'est révélée être un paramètre critique et un grand nombre de marqueurs des filaments d'actine ont été testés. 
Tout ces techniques sont présentées dans le chapitre suivant, décrivant les méthodes expérimentales, tandis que les résultats sont présentées dans les deux derniers chapitres. 
%\end{document}
