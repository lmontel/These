\chapter*{Remerciements}
Cette thèse représente près de quatre années de travail, depuis le début de mon stage de M2, jusqu'à la fin de la rédaction. Ce manuscrit couvre l'essentiel du travail scientifique accompli durant ce temps, mais seulement une partie de ce qui a fait ces années de thèse. Dans ces remerciements, je vais essayer de couvrir ce qui manque, les personnes, les rencontres, les discussions, qui ont rendu tout cela possible. 

Je vais sans doute en oublier, parce que dès qu'on essaie de faire une liste, il manque forcément au moins un élément. Toutes mes excuses d'avance à celui ou celle qui en fera les frais. 

Pour essayer d'être un peu rigoureuse, je vais procéder en remontant le temps. 

Il en suit logiquement que je remercie en premier lieu les membres du jury, pour avoir accepté d'évaluer mes travaux, pour être venus m'écouter parler de la petit pierre que je souhaite ajouter à l'immense édifice que forme aujourd'hui la recherche scientifique. 
J'espère que vous repartirez de ma soutenance avec une petite idée, un bout de savoir qui vous aura enrichi. 
Je remercie tout particulièrement les deux rapporteurs, qui ont eu la tâche plus ardue de lire entièrement les presque deux cents pages de cette thèse, pendant l'été qui plus est. Ils l'ont tous les deux faits avec rigueur et bienveillance, ce en quoi je leur suis très reconnaissante.

Toujours en chronologie inversée, je souhaite également remercier Tiana Jacquemart, stagiaire de M1 avec Sylvie pendant ma période de rédaction, et qui a bien voulu continuer les expériences que je ne pouvais plus faire faute de temps. Tiana, je te souhaite une bonne continuation pour la fin de ton Master, et je te remercie de ta patience avec moi lorsque la fin de thèse commençait à me stresser. 
L'été précédent, nous avions également accueilli une stagiaire de M1 qu'il me faut remercier, Marion Tesseydre, qui a réalisé avec moi les expériences sur les cellules fixées, puis les a poursuivies sans moi pendant que je partais en congé maternité.         

C'est au milieu de cette thèse qu'est née Valentine, pour chambouler toute ma vie. Rien de tel qu'un petit sourire plein de fossettes pour remettre en perspective les ratés expérimentaux ou les puzzles incompréhensibles de la thèse. Alors un grand merci à Valentine, pour sa nature souriante et sa curiosité sans limite, qui ont fait de "regarde, c'est beau" sa première phrase favorite. 

Par ailleurs, j'ai eu la chance de faire ma thèse dans un lieu privilégié, un laboratoire à l'ambiance conviviale, dans lequel ça me fera un pincement au cœur de ne plus venir travailler. Merci à Loïc de m'avoir accueilli au MSC, à Fabien mon homonyme, en particulier pour ses suggestions de jury et de post-docs, à Alain qui m'a tout appris de la culture cellulaire avec un humour pince-sans-rire, à Laurent qui fabriqué les pièces de mes montages expérimentaux, à François et Atef pour les discussions de mécanique cellulaire \dots
Merci à tous mes collègues thésards et post-docs, pour avoir mis une bonne ambiance dans le bureau, dans le labo et en conférences. Merci à Alessandra, avec qui j'ai fait mon premier Western Blot, et à qui je souhaite beaucoup de bonheur maintenant qu'elle va elle aussi avoir bientôt un petit bout à la maison. Merci à David, qui nous apprend à douter de tout, et à Sham pour les discussions complètement improbables en 767A. Merci également à Vicard, Loudjy, Laurie et Laura, d'avoir fait de ce bureau un lieu si agréable. 
Un grand merci à mes collègues de mécanique cellulaire, Élisabeth, Pauline, Nathalie et Kelly, pour toutes ces discussions intéressantes, à propos de mécanique ou non, et en particulier pour une certaine conférence en Autriche. J'envoie des remerciements tous particuliers à Élisabeth, que j'ai le plus côtoyé pendant ces années au labo, et qui est devenue une véritable amie. Tu me manques maintenant que tu es partie à l'autre bout du monde, mais je te souhaite bonne chance !

À l'origine de cette thèse, et donc de tout ça, il y a Sylvie, qui a accepté de me guider. Je n'arrive pas trop à savoir comment je pourrais te remercier de tout ce que tu m'as apporté pendant ces quatre années. J'ai côtoyé beaucoup de thésards, au laboratoire ou à l'extérieur, et je suis absolument convaincue d'être celle qui a eu la meilleur directrice de thèse. Tu as su me guider lorsque j'en avais besoin, tout en me laissant prendre les choses en main moi-même, me laisser explorer des territoires qui t'étaient inconnus comme la programmation et les bases de données, mais me suggérer au bon moment de changer d'approche lorsque les choses ne fonctionnaient pas.

En dehors du laboratoire, je tiens à remercier mon (grand) cercle d'amis, tous les participants des étés et hivers à Eygliers : mes collègues de promo de l'ENS Cachan Cécile, JB, Rémi Cornaggia, ou des promos d'avant, Daniel, Pascal, Sandrine, Chicco, ceux connus par Patrick, Clément et Sabrina, Rémi Bonnet, Camille, Floraine et Simon et ceux connus ensemble comme Guillaume, Chacha, Béa \dots et tout particulièrement Marie-Charlotte, avec qui on s'est forgées en tant qu'adultes, depuis maintenant 10 ans. 
Merci à tous pour ces pauses à la montagne, à jouer et à bien manger, qui permettent de s'aérer l'esprit, et de reprendre le moral au moment opportun. Merci également pour toutes les soirées et dîners chez nous, les parties de jeu de rôle, les sorties au parc et les séances de couture.

J'ai évidemment une immense dette envers Patrick, qui partage ma vie au quotidien, pour m'avoir moralement supportée, dans tous les sens du terme, pendant toutes ces années, lorsque j'avais l'impression de ne plus rien comprendre, et que rien ne fonctionnerait jamais. Ta patience est mon ancre dans la tempête. Merci d'être mon équipier dans l'aventure de la famille, et j'espère que la route sera très très longue. Merci également à Chantal et Ange, qui m'ont gentiment accueillie dans leur famille et dans leur maison. 

Merci à Claire et Julien, mes amis de toujours, avec qui j'ai grandi et découvert mes passions. Merci également à Valérian, Patrick et Claudine, pour toutes ces vacances passées ensemble. Vous êtes ma seconde famille. 


Enfin, merci à ceux sans qui rien de tout cela ne serait arrivé. Merci à mes parents d'avoir su développer et nourrir chez moi la curiosité, la soif de savoir, la persévérance, et l'ambition de faire de la recherche scientifique, et de m'en avoir donné les moyens. Merci à ma  s\oe ur d'avoir été toujours là (et d'avoir supporté les discussions scientifiques interminables à table), et d'avoir bien voulu devenir mon amie en plus de ma s\oe ur . 

Merci à tous. Vous avez tous apporté votre pierre dans l'édification de cette thèse, et une part vous en revient. 

