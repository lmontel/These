\documentclass{report}
\usepackage[T1]{fontenc}
\usepackage[utf8]{inputenc}
\usepackage[francais]{babel}
\usepackage{amsmath}
\usepackage{graphicx}
\usepackage[backend=biber,style=authoryear,bibencoding=utf8]{biblatex}
\usepackage[colorlinks,linkcolor=blue]{hyperref}

\addbibresource{biblio.bib}


%\begin{document}
\chapter{De la cellule au muscle}

Les muscles représentent environ 40\% de la masse totale d'un humain adulte. Ils participent à tous les phénomènes indispensables à notre vie : ils nous permettent de nous mouvoir, d'exercer des forces sur notre environnement, mais aussi de respirer, de faire circuler le sang, de digérer \dots 
Il existe trois principaux types de muscles : lisses, cardiaque, et squelettiques. Les muscle lisses, comme ceux que l'on trouve le long du tube digestif ou des vaisseaux sanguins, mais aussi dans la vessie ou l'utérus, ne peuvent pas être contrôlés volontairement. 
Le muscle cardiaque, présent uniquement au niveau du c\oe ur, a une organisation spécifique qui lui permet de maintenir des contractions régulières et permanentes, assez puissantes pour faire circuler le sang dans le système circulatoire. 
Les muscles squelettiques sont les seuls que nous contrôlons volontairement. Comme leur nom l'indique, ils s'ancrent au squelette par l'intermédiaire des tendons, et ce sont eux qui nous permettent de bouger nos membres.  
Il en existe environ 640, de toutes tailles, des minuscules muscles contrôlant les mouvements des yeux à l'énorme quadriceps.

\section{Organisation du muscle squelettique}

Dans le chapitre premier, une cellule typique et peu différenciée a été présentée, semblable à celle des précurseurs du muscle. Les cellules qui composent les muscles ont une organisation bien différente. 
En partant des cellules et des protéines qui ont été décrites dans les chapitres précédents, nous allons reconstituer l'architecture du muscle. 

\subsection{Du myoblaste au myocyte}

Les myoblastes sont les cellules progénitrices du muscles. Les C2C12 sont une lignée immortalisée de myoblastes murins, mais des myoblastes primaires peuvent également être mis en cultures \textit{in vitro}.
La différenciation des myoblastes se produit quand les myoblastes sont suffisamment nombreux et que les facteurs de croissance qui les maintiennent en prolifération viennent à manquer. 
Les myoblastes se mettent alors à sécréter de la fibronectine, la liaison intégrine/fibronectine étant indispensable à la différenciation.

Les myoblastes vont alors s'aligner, puis fusionner entre eux pour former des myotubes, qui sont des cellules plurinuclées et allongées. 
Les myoblastes étendent autour d'eux des lamellipodes et des filopodes pour contacter les cellules voisines. Au niveau de ces contacts, les compositions des lipides de la membrane change, on trouve un grand nombre de protéines comme les M-cadhérines, les intégrines et les filamines qui assurent un lien avec le cytosquelette. Ce dernier est réorganisé localement : un réseau dense se forme sous la membrane, mis sous tension par la myosine 2A. 
Le myoblaste envahit le myotube à l'aide de podosomes quasiment exclusivement composés d'actine en filaments, avant que des pores se forment entre les deux cellules pour achever la fusion. 

Une fois le myotube formé, il mature pour devenir une myofibre en développant une organisation spécifique de l'actine et de la myosine.


\subsection{Une organisation spécifique de l'actine}

À l'intérieur d'un myocyte totalement différencié, l'actine est organisée en unités appelées les sarcomères. 

Un sarcomère est composé de filaments d'actine et de filaments de myosines qui sont dotés de milliers de têtes de myosines. Le mouvement des têtes de myosines sur le filament d'actine fait coulisser les deux filaments l'un par rapport à l'autre, ce qui crée la contraction musculaire. 

Afin que les filaments d'actine, dont nous avons vu au chapitre 2 qu'ils sont ordinairement très dynamiques, maintiennent leur taille, ils sont coiffés aux deux extrémités par des protéines comme CapZ. 
Les nébulines, de très longues protéines qui peuvent se lier à environ 200 actines en filament, sont soupçonnées de servir d'étalon pour fixer la longueur des filaments d'actine dans les sarcomères. 

Autour des filaments d'actine, la tropomyosine, liée également à la troponine, régule la liaison entre le filament d'actine et le filament de myosine, en recouvrant ou découvrant les sites de liaisons des têtes de myosines à l'actine. 





\subsection{De la fibre au muscle}

\subsection{Différents types de muscles}

\section{La contraction musculaire}

\subsection{Au niveau du sarcomère}

\subsection{Force ou vélocité, il faut choisir}

