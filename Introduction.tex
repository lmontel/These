%\documentclass[10pt,twoside]{report}
%\usepackage[T1]{fontenc}
%\usepackage[utf8]{inputenc}
%\usepackage[francais]{babel}
%\usepackage{amsmath}
%\usepackage{graphicx}
%\graphicspath{{Figures/}}
%\usepackage[backend=biber,style=authoryear,bibencoding=utf8]{biblatex}
%\usepackage{fancyhdr}
%\pagestyle{fancy}
%\fancyhead[RO,LE]{}
%
%\usepackage[colorlinks,linkcolor=blue]{hyperref}
%\newcommand{\micro}{$\mathrm{\mu}$}
%\addbibresource{biblio2.bib}
%
%\begin{document}

\chapter*{Introduction et objectifs}
  Au moment où s'achève l'écriture de ce manuscrit, se déroule le Tour de France cycliste. En observant les jambes des coureurs, il est parfois difficile de se dire qu'elles sont formées sur la même base que les nôtres. Si l'on peut se douter qu'un champion doit être doté d'une conjonction de caractéristiques génétiques particulières et d'années de travail, il demeure que la plupart d'entre nous, même sans don particulier, peut atteindre des performances sportives étonnantes après un entraînement rigoureux. C'est que notre corps est adaptable, et en particulier nos muscles, qui en réponse à des sollicitations répétées, vont devenir plus forts, plus rapides, plus précis. La musculation constitue l'exemple le plus immédiat et quotidien de la mécanotransduction, cette capacité qu'a notre corps d'interpréter de manière biologique un signal mécanique, et d'y apporter une réponse. Cette capacité se retrouve à toutes les échelles de l'organisme, de l'organe à la cellule qui le compose. 
  
  Cette thèse s'intéresse tout particulièrement à ce phénomène de mécanotransduction à l'échelle de la cellule musculaire. Les cellules d'un muscle en exercice sont soumises à des forces extérieures, elles vont y réagir en formant plus de masse musculaire. Ce champ de recherche se place alors naturellement à la rencontre de la physique, qui étudie particulièrement les forces mécaniques, et de la biologie cellulaire. 
  
  Le premier chapitre de cette thèse sera consacré à la description rapide de la cellule animale, des éléments qui la composent et de leur fonctionnement. 

	Les cellules maintiennent leur forme, se déplacent, se divisent grâce à un réseau de filaments appelé cytosquelette. Ce cytosquelette est essentiel dans la réponse des cellules aux forces extérieures, et détermine en grande partie leur réponse mécanique. L'actine est une protéine du cytosquelette, présente en grande quantité dans les cellules musculaires, qui est à la fois un acteur majeur des propriétés rhéologiques de la cellule et de la contraction musculaire. Elle peut former des filaments semi-rigides, dans une structure dynamique toujours en construction et déconstruction, formant un équilibre entre l'actine seule (en monomères) et l'actine incorporée dans les filaments.  Le deuxième chapitre sera consacré à la description de l'actine et des structures qu'elle forme dans la cellule. 

  À l'aide d'études sur des souris, l'équipe d'Athanassia Sotiropoulos a mis en évidence le rôle d'une voie de signalisation dans l'hypertrophie et l'atrophie des muscles en réponse à une contrainte mécanique. Le facteur de transcription Serum Response Factor, qui contrôle les gènes du cytosquelette et de la différenciation musculaire, et son cofacteur Myocardin-Related Transcription Factor, sont les deux acteurs centraux de ce phénomène. Or il a déjà été montré que MRTF-A est régulée par l'actine. Ensemble ces deux protéines font le pont entre la sensibilité mécanique de la cellule, au niveau du cytosquelette d'actine, et l'activation des gènes par SRF. Le troisième chapitre de cette thèse expliquera en détail le rôle de MRTF-A, et la manière dont cette protéine interagit avec l'actine pour induire une réponse génétique à un stimulus mécanique. 
  
  Dans les cellules musculaires, l'actine est organisée d'une manière extrêmement spécifique, pour permettre la contraction musculaire. Cette organisation sera décrite dans le chapitre 4. 
  
  Enfin, le chapitre 5 sera consacré aux propriétés mécaniques des cellules et du cytosquelette d'actine. Il décrira les outils que les physiciens ont élaboré pour exercer et mesurer des forces à des échelles cellulaire et sub-cellulaire, et les lois mécaniques mises en évidence par ces mesures. 
  
  Ces cinq premiers chapitres dessinent tout le chemin qui se trouve entre le signal mécanique que va sentir la cellule, et la réponse génétique qu'elle va y apporter : une contrainte mécanique peut être à l'origine d'une réorganisation du cytosquelette d'actine, les modifications de ce dernier vont réguler MRTF-A, qui va à son tour contrôler le facteur de transcription SRF, et déclencher l'expression d'un programme génétique. Plus particulièrement, MRTF-A doit être dans le noyau de la cellule pour pouvoir activer le programme génétique avec SRF. Or la localisation de MRTF-A dans la cellule est régulée par la quantité de monomères d'actine disponible. Lorsqu'il y a une grande quantité de monomères, ceux-ci se lient à MRTF-A et l'empêchent d'aller dans le noyau. Au contraire, lorsque les monomères viennent à manquer, MRTF-A peut être importée dans le noyau et y interagir avec SRF. 
  
  L'équipe dans laquelle j'étais en thèse avait déjà constaté auparavant que l'application de contraintes mécaniques répétées sur une cellule musculaire induit une polymérisation du cytosquelette d'actine en réponse. Cela, associé aux résultats de l'équipe de l'Institut Cochin, conduit à formuler l'hypothèse suivante : lorsque les cellules musculaires sont soumises à des forces extérieures, leur cytosquelette d'actine se renforce en polymérisant, cette polymérisation d'actine conduit à un manque de monomères pour se lier à MRTF-A, qui est alors libre d'aller dans le noyau se lier à SRF, pour activer le programme génétique de développement musculaire. 
  
  L'objectif de ce travail est d'étudier comment les signaux mécaniques vont réguler MRTF-A à travers les modifications du cytosquelette d'actine. Pour cela, la première étape a été de construire des outils nous permettant d'exercer des forces à l'échelle de la cellule unique, et de mesurer leur réponse mécanique et biologique. Les pinces magnétiques ont permis de caractériser la rhéologie des cellules musculaires, et d'observer la réponse biologique de MRTF-A. L'étireur de cellules a permis d'étudier de manière systématique sur une grande population de cellules la réponse du système à des déformations imposées. Les techniques de microscopie de fluorescence ont été utilisées pour visualiser les deux protéines, actine et MRTF-A, à l'intérieur de la cellule. Toutes ces méthodes expérimentales sont décrites en détail dans le chapitre 6. 
  
  À l'aide de ces outils, nous avons pu caractériser les propriétés mécaniques de myoblastes, qui seront présentées au chapitre 7. 
  
  Enfin, au chapitre 8, nous montrerons que l'application de contraintes mécaniques peut induire une polymérisation ou une dépolymérisation du cytosquelette d'actine, conduisant à des changements de localisation de MRTF-A dans les cellules. Nous verrons également que l'utilisation de marqueurs du cytosquelette perturbe l'équilibre entre les filaments d'actine et les monomères d'une manière qui est toujours perceptible dans la localisation de MRTF-A. 





%\end{document}